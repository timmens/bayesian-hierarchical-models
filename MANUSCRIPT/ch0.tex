\section{Introduction}

The advent of extensive data collection and data storage has lead practitioners of applied econometrics to consider bigger and more complex models.
However, with a lack of simultaneous bursts of growth in (economic) theory, users often find themselves employing purely data-driven approaches.
This is not bad per se, nonetheless, without imposing structure it is hard to derive non-trivial inferences from such analyses.
In this paper we consider a common type of data which is inherently equipped with certain structure and present how to utilize this additional level of information.
Moreover we focus on so called Bayesian approaches, which allow for extra knowledge to be readily added to the model.

In section \ref{sec:bayesian_thinking} we introduce the formal notion of Bayesian thinking and estimation, which we expose by solving a simple normal model.
We end the section by presenting the idea behind Markov chain Monte Carlo, as it is a fundamental concept in modern Bayesian statistics.

In section \ref{sec:hierachical_modeling} we present hierachical data ---which contains aforementioned additional structure--- and ways of modeling it.
Our theoretic considerations end with a presentation of hierachical linear models, the class of models we use in the remaining parts of the paper.

In this section \ref{sec:application} we discuss an application of the Baysian hierarchical modelling approach in the form of a literature review. We then use real world data to give a practical illustration of the work we have developed in the previous sections of our paper and compare that to a frequentist approach to analyzing multilevel data with the aim of showcasing what differences/similarities there are between the two methods. We then depart form the comparative analysis to look deeper into the Bayesian approach and conduct robustness checks. We close the section off by a discussion of the limitations and challenges encountered in this section.

We note that the parts \hyperref[sec:bayesian_thinking]{\emph{Bayesian Thinking and Estimation}} and \hyperref[sec:hierachical_modeling]{\emph{Hierachical Models}} were written by Tim Mensinger, while the part \emph{Monte Carlo Study} was written by Markus Schick and the part \emph{Application} by Linda Maokomatanda.
