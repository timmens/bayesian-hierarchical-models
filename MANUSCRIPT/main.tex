\documentclass[a4paper,11.5pt]{article}

\usepackage{preamble}

\title{\textbf{Bayesian Hierarchical Models}\\
\large Research Module - Econometrics and Statistics - 2019/2020
}
\date{}
\author{Linda Maokomatanda%
  \thanks{\href{mailto:linda.maokomatanda@uni-bonn.de}{linda.maokomatanda[at]uni-bonn.de}}}
\author{Tim Mensinger%
  \thanks{\href{mailto:tim.mensinger@uni-bonn.de}{tim.mensinger[at]uni-bonn.de}}}
\author{Markus Schick%
  \thanks{\href{mailto:markus.schick@uni-bonn.de}{markus[at]email.de}}}
\affil{University of Bonn}

\begin{document}
%\setlength{\parfillskip}{0pt plus 2fil}
\setlength{\abovedisplayskip}{-20pt}
\setlength{\belowdisplayskip}{0pt}
\setlength{\abovedisplayshortskip}{0pt}
\setlength{\belowdisplayshortskip}{0pt}
\onehalfspacing

\thispagestyle{empty}
\pagenumbering{gobble}
\maketitle
\begin{abstract}
  In this paper we introduce the core topics of Bayesian statistics with a focus on modern sampling techniques.
  We then present hierarchical models and show how they can be seen as a natural extension to the Bayesian prior design.
  In our simulation study we show how a simple Bayesian hierarchical model behaves under varying sample sizes. We further test how correctly and wrongly specified priors as well as noninformative priors influence the results.
  At last we provide a small literature review on hierarchical models and fit a standard hierarchical linear model in an educational setting. We supply all code to reproduce our results on our project page.\footnote{\url{https://github.com/timmens/bayesian-hierarchical-models}}
\end{abstract}

\newpage
\setcounter{tocdepth}{2}
\tableofcontents
\newpage

\pagestyle{plain}
\pagenumbering{arabic}

\section{Introduction}

The advent of extensive data collection and data storage has lead practitioners of applied econometrics to consider bigger and more complex models.
However, with a lack of simultaneous bursts of growth in (economic) theory, users often find themselves employing purely data-driven approaches.
This is not bad per se, nonetheless, without imposing structure it is hard to derive non-trivial inferences from such analyses.
In this paper we consider a common type of data which is inherently equipped with certain structure and present how to utilize this additional level of information.
Moreover we focus on so called Bayesian approaches, which allow for extra knowledge to be readily added to the model.

In section \ref{sec:bayesian_thinking} we introduce the formal notion of Bayesian thinking and estimation, which we expose by solving a simple normal model.
We end the section by presenting the idea behind Markov chain Monte Carlo, as it is a fundamental concept in modern Bayesian statistics.

In section \ref{sec:hierachical_modeling} we present hierachical data ---which contains aforementioned additional structure--- and ways of modeling it.
Our theoretic considerations end with a presentation of hierachical linear models, the class of models we use in the remaining parts of the paper.

In this section \ref{sec:application} we discuss an application of the Baysian hierarchical modelling approach in the form of a literature review. We then use real world data to give a practical illustration of the work we have developed in the previous sections of our paper and compare that to a frequentist approach to analyzing multilevel data with the aim of showcasing what differences/similarities there are between the two methods. We then depart form the comparative analysis to look deeper into the Bayesian approach and conduct robustness checks. We close the section off by a discussion of the limitations and challenges encountered in this section.

We note that the parts \hyperref[sec:bayesian_thinking]{\emph{Bayesian Thinking and Estimation}} and \hyperref[sec:hierachical_modeling]{\emph{Hierachical Models}} were written by Tim Mensinger, while the part \emph{Monte Carlo Study} was written by Markus Schick and the part \emph{Application} by Linda Maokomatanda.

\section{Bayesian Thinking and Estimation}
In this section we will introduce the core topics of Bayesian data analysis, and whenever possible compare proposed methods and results to their frequentist counterpart.
As we will see, Bayesian statistics differs from mainstream statistics on a fundamental level; Thus, we have to start there.

Before beginning let us talk shortly about our notation.
We try to use a standard notation whenever possible, nonetheless, we make one exception in that we write $p(X)$ for the probability density function of the random variable $X$, where $X$ might be scalar-valued or vector-valued.
In cases where this leads to confusion we introduce a subscript as $p_X(\cdot)$.

\subsection{Probabilistic Modeling}
Before going further, let us first formalize our notion of stochastic modeling.
Imagine being interested in some \textit{phenomena}, for example the effect of bigger class sizes on school children performance.
Most phenomena cannot be observed directly and only manifest themself through some latent variable system.
We can still hope to learn about the phenomena by studying the observational process around it.
Clearly the way in which a phenomena reveals itself is dependent on its environment;
The observed effects for school-children in sub-saharan africa might look very different to the ones in central europe.
To derive sensible results from our analysis we need to postulate the existence of a \textit{true data generating process}, which captures the way in which the observational process adheres to the effects of the phenomena of interest given its environment.
We define this object as a probability distribution $p_0$ on the observational space $\mathbb{Z}$.
In this step we move from a model to a probabilistic model, in that we allow our system of interest to be influenced by randomness and not only be characterized by a deterministic process.
In practice $p_0$ is rarely known, therefore the challenge lies in recovering a distribution using the observed data which is as close as possible to $p_0$.
This is usually done by assuming that the true data generating process falls in some class of models, for example a linear model with normal errors.
Formally, we restrict our attention to a subset of potential observational processes $\mathbb{M}$ over the whole space of distributions on $\mathbb{Z}$.
The beauty of using a model class approach in the construction of $\mathbb{M}$ is that for each distribution $p \in \mathbb{M}$, we can find a parameterization $\theta$ in the configuration space $\Theta$; For example, the class of multivariate normal distributions is parameterized by its mean and covariance $(\mu, \Sigma) = \theta \in \Theta$.
The goal of all subsequent statistical analysis is then to utilize the observed data to determine the regions in $\Theta$ which are most consistent with $p_0$ and simultaneously to capture our uncertainty about these statements.
If $p_0 \in \mathbb{M}$ we can find a parameterization $\theta_0$ which corresponds to the true data generating process; naturally we seek estimators that determine regions close to $\theta_0$. If, however, $p_0 \not\in \mathbb{M}$ we enter the world of model misspecification which leads to all sorts of problems.
For everything that follows let us therefore make the omnipresent assumption that $p_0 \in \mathbb{M}$.

\subsection{Schools of Thought}
Next we discuss how the Bayesian and the classical mindset differ. In particular we focus on the predominant way of thinking for most of statistical history, \textit{frequentist statistics}. We do not aim at an exhaustive overview here nor do we presume that the individual statistician belongs to one and only one of the following categories.

\paragraph{Frequentist.} The frequentist approach assumes that the true data generating process is completely specified by an unkown but fixed quantity $\theta_0 \in \Theta$.
For a sensible comparison we focus on approaches where the \emph{likelihood function} $p(z; \theta)$ is defined either explicitly or implicitly during the modeling process, as for example in maximum-likelihood estimation.
The main challenges include finding a (point) estimator for $\theta_0$, quantifying the uncertainty of the estimate and testing hypothesis.
One fundamental idea which stretches over all these topics is the interpretation of probability as the limit of an infinite sequence of relative frequencies ---hence the name.
That is, the probability of an event happening is just the limit of the frequency of that event happening over infinitely many independent experiments.
What are the implications of this understanding of probability?
Many interesting questions do not provide us with a thought experiment in which we can consider an ever increasing sequence of experiments.
In these cases using probability is either trivial or lacking an indisputable interpretation.
As we use the mathematical rigor of probability theory in our formal derivations, we will obtain (mathematically) correct results; however, the interpretation of these results might be highly unintuitive ---a known example is provided by the interpreation of confidence intervals; is the true value contained in the given confidence interval? We do not know, but were we to repeat the analysis with newly drawn data infinitely often, the true value would be in the interval 95\% of the time.
Since $\theta_0$ is fixed all probability statements regarding this object are trivial, that is, either one or zero. This propagates to the problem of hypothesis testing.
All hypothesis are either true or false and therefore have probabilities of one or zero.
A hypothesis $H_0$ is rejected if conditional on $H_0$ being true the probability of observing the data in the given sample is lower than some threshold, i.e. $P(\text{data} \mid H_0) < \alpha$.
Note that this statement does not tell us anything about $H_0$ directly, but only about the data at hand.

\paragraph{Bayesian.} The Bayesian approach also assumes that there may be a true data generating process specified by some (maybe fixed) quantity $\theta_0 \in \Theta$.
The main difference is their understanding of probability as a subjective quantification of uncertainty.
In this view one is not limited to assigning non trivial probability statements only to objects that appear random in sequential experiments.
We can see the direct utility of this liberation by considering a special case of Bayes theorem
\begin{align}
  p(\theta \mid \text{data}) = \frac{p(\text{data} \mid \theta) p(\theta)}{p(\text{data})} \propto p(\text{data} \mid \theta) p(\theta) \,,
\end{align}
which reads
\begin{align}
  posterior = \frac{likelihood \times prior}{evidence} \propto likelihood \times prior \,.
\end{align}
In the frequentist setting this is of no use, since the statement $p(\theta)$ is nonsensical ---remember that probabilistic statements about fixed quantities are meaningless from a frequentist perspective.
This already outlines the main criticism of Bayesian analysis: Where does the prior $p(\theta)$ come from?
With the scientific goal of objectivity in mind, many feel uneasy with results being dependent on a subjective choice of a prior.
In what follows we will embark on the Bayesian idea without providing much more fundamental criticism; nonetheless, when adequate we will consider the influence of different priors on the posterior.
To end this comparison, what then are the main tasks associated with a Bayesian analysis?
These can be categorized by (i) obtaining the posterior distribution (or something equivalent) and (ii) communicating the information held in the posterior.
The first consists of defining the likelihood and (additionally to the frequentist approach) constructing a prior distribution and afterwards combining those to compute the posterior.
This computation can sometimes be achieved analytically, but in most cases one has to rely on algorithms to obtain samples of the posterior.
The second part consists of plotting the marginal posterior distributions, computing expectations of the form $\Exp{h(\theta) \mid \text{data}}$ and testing hypothesis.
All of the above can be done independent of the posterior being available analytically or through samples.
A clear difference can be seen when considering hypothesis testing.
Sacrificing \textit{objectivity} allows us to answer the questions we usually want to ask:
\begin{align}
  P(H_0: \theta \in S \mid \text{data}) = \int_{\theta \in S} p(\theta \mid \text{data}) \mathrm{d}\theta \,.
\end{align}
In the subsequent paragraphs we will be mostly occupied with the first category, computing the posterior, with occasional remarks on the second; in particular, the approximation of expectations.

\subsection{Solving for the posterior analytically}
In this subsection we present the analytical derivation of the posterior distribution of mean and variance parameters in a univariate normal model for two priors.
We will compare the results to the appropriate classical method, namely maximum likelihood.

In both cases, let us assume that we observe an iid sample $y = (y_1, \mydots, y_n)$
with $y_i \sim \normal{\mu, \sigma^2}$.
Our interest lies in solving for the marginal posteriors $p(\mu \mid y)$ and $p(\sigma^2 \mid y)$.

To be precise, in a full bayesian analysis we assume $\theta = (\mu, \sigma^2)$ to be a random quantity, thus the correct statement should be $y_i \mid \theta \sim \normal{\mu, \sigma^2}$.
In situations where this conditional dependence is clear many writers will use the first notation.
Here we try to be as pedantic as possible to avoid any confusion and will therefore stick to the second notation.

As it will be of major importance in the subsequent sections we remind the reader of some probability distributions uncommon in the non-Bayesian world.

\begin{definition}{(Scaled inverse $\chi^2$ distribution).}
  Let $\nu > 0$ and $\tau^2 > 0$ be parameters representing degrees of freedom and scale, respectively. The family of \emph{scaled inverse $\chi^2$ distributions} is characterized by its probability density function, namely
  \begin{align}
    p(x) \propto x^{-(1 + \nu / 2)} \EXP{\frac{-\nu \tau^2}{2 x}} \quad \text{for} \, x \in (0, \infty) \,,
  \end{align}
  where the constant of integration is ignored for clarity.
  We write $X \sim \scaledInvChi{\nu, \tau^2}$ to denote that the random variable $X$ follows a scaled inverse $\chi^2$ distribution with parameters $\nu$ and $\tau^2$.
\end{definition}

\subsubsection*{Uninformative Prior}
We start our first Bayesian analysis by considering a prior which contains virtually no information.
This results in an analysis being mainly, if not completely, driven by the likelihood.
A common assumption is independence of the individual priors, that is $p(\theta) = p(\mu, \sigma^2) = p(\mu) p(\sigma^2)$.
A natural choice of declaring full ignorance of prior information is to assign a prior over the complete domain of the random parameter. For our case this mean $p(\mu) \propto 1$.
We note that this does not define a proper proability distribution, which will not matter in this case but can lead to problems in others; see for example section 4.2 in \citet{kass1996}.
Since the variance is restricted to be positive we impose a uniform prior on the log-transform thereof: $p(\log \sigma) \propto 1$.
Using that $x \mapsto \text{exp}^2(x)$ is one-to-one we can apply standard methods, which yield the density of the transformed variable
\begin{align}
  p(\mu, \sigma^2) \propto (\sigma^2)^{-1} \,.
\end{align}
As usual the likelihood is given by
\begin{align}
  p(y \mid \mu, \sigma^2) \propto (\sigma^2)^{-n/2} \EXP{-\frac{1}{2\sigma^2}\sum_i (y_i - \mu)^2} \,,
\end{align}
where we dropped all proportionality constants.
Using the above we can apply Bayes theorem to yield
\begin{align}
  p(\mu, \sigma^2 \mid y) &\propto p(y \mid \mu, \sigma^2) p(\mu, \sigma^2)\\
  &\propto (\sigma^2)^{-(n+2)/2} \EXP{-\frac{1}{2\sigma^2}\sum_i (y_i - \mu)^2} \,.
\end{align}
From here we can derive the marginals by integrating out the respective other parameter. This is formalized in the following two proposition.

\begin{proposition}
  Under the uniform prior from above we find
  \begin{align}
    \mu \mid y &\sim t_{n-1}(\bar{y}, s^2/n) \,,\\
    \sigma^2 \mid y &\sim \scaledInvChi{n-1, s^2} \,,
  \end{align}
  where $s^2 = \frac{1}{n-1} \sum_i (y_i - \bar{y})^2$ denotes the sample variance and $\bar{y} = \frac{1}{n} \sum_i y_i$ the sample mean.
\end{proposition}
\begin{proof}
See appendix.
\end{proof}

Having derived the marginal posterior distributions, we can compare the results to their maximum likelihood (ML) counterpart.
Since the ML estimates $\left(\argmax{\theta \in \Theta}p(y \mid \theta)\right)$ are point estimates, we consider the similar \emph{maximum a posteriori} (MAP) estimate $\left(\argmax{\theta \in \Theta} p(\theta \mid y)\right)$, as well as the posterior mean and variance. All results are summarized in table \ref{table:comp_uniform_bay_ml}.

\begin{table}[ht]
\centering
\def\arraystretch{1.3}
{\small
 \begin{tabular}{c | c c c c c}
 Parameter & ML Estimate & ML Variance & MAP & Posterior Mean & Posterior Variance\\[0.5ex]
 \hline
 $\mu$ & $\bar{y}$ & $\sigma^2/n$ & $\bar{y}$ & $\bar{y}$ & $s^2 / n$\\
 $\sigma^2$ & $\frac{n-1}{n} s^2$ & $2 \sigma^4 /n$ & $\frac{n-1}{n+1} s^2$ & $\frac{n-1}{n-3} s^2$ & $\frac{2 (n-1)^2}{(n-3)^2 (n-5)} s^4$\\
 \end{tabular}
 }
\caption{\small {Comparison of Bayesian estimates using an uninformative prior and ML estimates. See appendix for a detailed derivation.}}
\label{table:comp_uniform_bay_ml}
\end{table}


\subsubsection*{Conjugate Prior}
We have seen that using an uninformative prior leads to results that are very similar to the ones obtained by a ML approach.
In case information on the parameters is available prior to observing the data we can utilize this fact by properly modeling the prior distribution.
Since we are interested in analytical results in this section,b we cannot mix any prior with any likelihood, as the product might not be of known form.
This leads us to the class of \emph{conjugate priors}.

\begin{definition}{(Conjugate prior).}
Let the likelihood $p(y \mid \theta)$ be given and assume that the prior distribution $p(\theta)$ is a member of some family $\mathcal{F}$ of probability distributions.
We say that $p(\theta)$ is a \emph{conjugate prior} if the posterior $p(\theta \mid y)$ is also a member of $\mathcal{F}$.
\end{definition}

Conjugate priors were of particular importance in the early stages of Bayesian statistics since these give the practitioner certainty that the posterior follows a distribution which is known and computable. Moreover, nowadays we still see conjugate priors in use as they allow for a full or partial analytical derivation, which increases the accuracy of results or shortens the runtime of programs. For more complex models however conjugate priors can become too restrictive. We discuss solutions to this problem in the next section.

Consider again the likelihood but written to demonstrate its dependence on $\mu$ and $\sigma^2$
\begin{align}
  p(y \mid \mu, \sigma^2) \propto (\sigma^2)^{n/2} \EXP{-\frac{1}{2\sigma^2} n\left[ (\mu - \bar{y})^2 + (\bar{y^2} - \bar{y})^2 \right]} \,.
  \label{eq:likelihood2}
\end{align}
We want to construct a two dimensional prior for $(\mu, \sigma^2)$.
A theme to which we will be coming back is that modeling higher dimensional parameters by modeling many lower dimensional (sub)parameters using conditioning is often easier than modeling the complete distribution.
Here we utilize the equality $p(\mu, \sigma^2) = p(\mu \mid \sigma^2) p(\sigma^2)$.
By looking at the likelihood (equation \ref{eq:likelihood2}) we note that in order to \emph{not} change the inherent structual dependence on the parameters, $\mu \mid \sigma^2$ has to be distributed according to $\normal{\mu_0, \sigma^2 / \kappa_0}$ with so called \emph{hyperparameters} $\mu_0$ and $\kappa_0 > 0$.
Similary, we note that an informative prior for $\sigma^2$ has to respect the structure in which $\sigma^2$ appears in the likelihood.
We achieve this when $\sigma^2 \sim \scaledInvChi{\nu_0, \sigma_0^2}$ with hyperparameters $\nu_0$ and $\sigma_0^2 > 0$.
Following \citet{gelmanbda04} we write $(\mu, \sigma^2) \sim \NormalscaledInvChi{\mu_0, \sigma_0^2 / \kappa_0; \nu_0, \sigma_0^2}$ with corresponding density function
\begin{align}
  p(\mu, \sigma^2) = p(\mu \mid \sigma^2) p(\sigma^2) \propto (\sigma^2)^{\frac{3 + \nu_0}{2}} \EXP{-\frac{1}{2\sigma^2} \left[\nu_0 \sigma_0^2 + \kappa_0(\mu_0 - \mu)^2 \right]} \,.
\end{align}

Multiplying the likelihood with our constructed prior we get the joint posterior (up to an integration constant)
\begin{align}
  p(\mu, \sigma^2 \mid y) \propto& (\sigma^2)^{-\frac{3 + n + \nu_0}{2}} \times\\
  & \times \EXP{-\frac{1}{2 \sigma^2} \left[ (\mu - \bar{y})^2 + (\bar{y^2} - \bar{y})^2 + \nu_0\sigma_0^2 + \kappa_0(\mu - \mu_0)^2 \right]} \,.
  \label{eq:conjugate_posterior}
\end{align}

\begin{proposition}
  The posterior distribution of $(\mu, \sigma^2) \mid y$, as given by the conditional density in equation \ref{eq:conjugate_posterior}, is $\NormalscaledInvChi{\mu_n, \sigma_n^2/\kappa_n; \nu_n, \sigma_n^2}$ distributed, where
  \begin{align*}
    \nu_n &= \nu_0 + n \,; \quad \kappa_n = \kappa_0 + n \,; \quad \mu_n =\frac{\kappa_0}{\kappa_0 + n}\mu_0 + \frac{n}{\kappa_0 + n}\bar{y} \,,\\
    \sigma_n^2 &= \left[\nu_0 \sigma_0^2 + (n-1)s^2 + \frac{\kappa_0 n}{\kappa_0 + n} (\bar{y} - \mu_0)^2\right] /\nu_n \,.
  \end{align*}
  \label{prop:posterior_conjugate}
\end{proposition}
\begin{proof}
  See appendix.
\end{proof}

Since the prior and the posterior are both normal scaled inverse $\chi^2$ distributed, we can speak of a conjugate prior.
Using the intermediate finding from proposition \ref{prop:posterior_conjugate} we can derive the main result of this section.

\begin{proposition}
  The marginal posterior distributions are given by
  \begin{align*}
    \mu \mid y &\sim t_{\nu_n}(\mu_n, \sigma_n^2 / \kappa_n)\\
    \sigma^2 \mid y &\sim \scaledInvChi{\nu_n, \sigma_n^2}\,,
  \end{align*}
  where $\nu_n, \sigma_n^2, \mu_n$ and $\kappa_n$ are as in proposition \ref{prop:posterior_conjugate}.
\end{proposition}
\begin{proof}
  See appendix.
\end{proof}

\begin{table}[ht]
\centering
\def\arraystretch{1.3}
{\small
 \begin{tabular}{c | c c c c c}
 Parameter & ML Estimate & ML Variance & MAP & Posterior Mean & Posterior Variance\\
 \hline
 $\mu$ & $\bar{y}$ & $\sigma^2 / n$ & $\mu_n$ & $\mu_n$ & $\sigma_n^2 / \kappa_n$\\
 $\sigma^2$ & $\frac{n-1}{n}s^2$ & $2 \sigma^4 / n$ & $\frac{\nu_n}{\nu_n + 2} \sigma_n^2$ & $\frac{\nu_n}{\nu_n - 2} \sigma_n^2$ & $\frac{2 \nu_n^2}{(\nu_n - 2)^2(\nu_n - 4)} \sigma_n^4$
 \end{tabular}
 }
\caption{{\small Comparison of Bayesian estimates using a conjugate prios and ML estimates.}}
\label{table:comp_conjugate_bay_ml}
\end{table}


Let us first consider the parameter $\mu$.
We note that the posterior mean (and MAP) is given by $\mu_n =\frac{\kappa_0}{\kappa_0 + n}\mu_0 + \frac{n}{\kappa_0 + n}\bar{y}$, which forms a convex combination of the prior $\mu_0$ and the sample average $\bar{y}$, with weights given by the sample size and $\kappa_0$.
For any fixed $n$ this pulls our estimate of the posterior mean away from $\bar{y}$ and closer to $\mu_0$ (and vice versa).
Further, we can use the hyperparameter $\kappa_0$ to express our uncertainty in $\mu_0$ (or $\bar{y}$ for that matter).
Rewritting the posterior variance using the \emph{Laundau notation} we get $\sigma_n^2 / \kappa_n = \frac{n-1}{\nu_n \kappa_n} s^2 + \mathcal{O}(\frac{1}{\nu_n\kappa_n}) = \frac{n}{(\nu_0 + n)(\kappa_0 + n)} s^2 + \mathcal{O}(1/n^2)$.
As the sample size $n$ grows the information contained in the likelihood should dominate the prior.
We observe this phenomena as the approximate asymptotic behavior of the posterior resembles that of the maximum likelihood estimator.
First, as $n$ tends to infinity $\nu_n = \nu_0 + n$ tends to infinity and the t distribution becomes indistinguishable from a normal.
Second, as $n$ grows the posterior mean is dominated by $\bar{y}$. And at last, for large $n$ the posterior variance is accurately approximated by $\sigma^2 / n$.
We refrain from an analogous analysis for $\sigma^2$ and only note that similar results hold, as can be seen in table \ref{table:comp_conjugate_bay_ml}.

What is gained from using an informative prior here?
Using the conjugate prior from above we have four hyperparameters at hand to model our prior knowledge about the parameters.
These can be used to represent very detailed to very vague information.
In any case, we were able to see that as we collect more and more data, the likelihood dominates our results.
A clear advantage of an analytical derivation is that we know exactly how the prior influences the posterior.
However, we have also seen that even for this \emph{very} simple model, the derivation is far from obvious.
As we consider more complex models using more parameters we have to make more restrictive assumptions on the way we model our prior information, if an analytical analysis is even possible.
For this reason among others, in the next section we present a method which trades off the clarity of an analytical result for the generality of being able to combine near arbitrary priors with complex, possibly high-dimensional likelihoods.


\newpage
\subsection{Sampling From The Posterior}
In this section we consider approaches that allow us to characterize the posterior distribution in complex settings using sampling methods.

For the rest of this section let us assume that we observe data $z \in \mathbb{Z}$ and can compute the likelihood $p(z \mid \theta)$ and prior $p(\theta)$ for $\theta \in \Theta$.
As before, our goal lies in analyzing the posterior distribution given by $p(\theta \mid z) \propto p(z \mid \theta) p(\theta)$.
Unlike before however, we now consider cases where the posterior is highly complex or even non-existent in analytical form, which happens for example when the likelihood contribution stems from an algorithmic computational model.

Say we are somehow able to draw independent samples $\theta_1, \mydots, \theta_n$ from $p(\theta \mid z)$.
By independence we get the well known result $\frac{1}{\sqrt{n}} \sum_i h(\theta_i) \overset{d}{\longrightarrow} \mathcal{N} \left( \Exp{h(\theta) \mid z}, \var{h(\theta) \mid z} \right)$, under mild conditions on $h$ and $p(\theta \mid z)$.
As we are able to formulate many quantities of interest using expectations ---probability statements can be written as expectations--- and as we can approximate percentiles from a (large) sample, we should be able to adequately summarize the posterior distribution if we are able to draw (independent) samples from it.

In the subsequent paragraphs we will discuss efficient methods to sample from the posterior, even if we cannot compute the integration constant $\int p(z \mid \theta) p(\theta) \mathrm{d}\theta$.
We will see that these methods do \emph{not} produce independent but autocorrelated samples.
With this in mind, we follow the creational process of these methods and first state the assumptions which have to be satisfied by the sampling process in order to yield good properties as for example a central limit theorem for dependent samples.
Then we present the \emph{Metropolis-Hastings algorithm}, which creates samples that fulfill the above criteria.
At last we talk about cases in which the Metropolis-Hastings algorithm fails and what can be done instead.

\paragraph{Markov Chain Monte Carlo.}
Say we are able to construct a \emph{Markov chain} with unique invariant distribution equal to the posterior distribution we want to sample from.
Given we know the transition kernel, Markov chains are very easy to simulate.
Hence, we could start a chain, let it run \emph{long enough} and at some point consider all subsequent realizations as draws from the posterior; this is the core idea of MCMC ---we defer questions regarding the creation of transition kernels which result in specific invariant distributions until next paragraph.
In practice we never know for sure when a chain is run \emph{long enough}.
In part 3 we present some measures that help during the application.
Under mild conditions we can get something similar to a law of large numbers for Markov chains (see e.g. \citet{roberts2004}, Fact 5).
This tells us that if we run the chain forever, our average will eventually converge to the number we seek.
However, forever is a very long time.
That is why we focus on assumptions which admit a central limit theorem with the usual $\sqrt{n}$ convergence rate, as it allows us to make more rigorous statements about our confidence in the whereabouts of the estimator for large but finite samples.

\begin{remark}
  As is often the case, there are many different sets of assumptions that allow for a CLT.
  The following theorem presents a particular set of assumptions which will be seen to have favorable properties when also considering the creation process.
  We remark that we will \emph{not} formally introduce all concepts and will provide only a heuristic explaination of the assumptions.
  This is due to the fact that Markov chain theory on general state spaces requires a good understanding of measure theory which we do not want to assume as a prerequisite.
  The interested reader is refered to \citet{roberts2004} for a survey on recent advances with application to MCMC and to \citet{meynandtweedie09} for a comprehensive treatment of Markov chain theory.
\end{remark}

\begin{theorem}{(A Central Limit Theorem for Markov Chains).}\label{thm:mcclt}
  Let $\{X_n \}$ be a (discrete time) Markov chain and $\pi$ a probability distribution on the same space.
  Consider some measurable function $h$ with $\Expwrt{\pi}{h^2} < \infty$.
  Define $\sigma^2(h) := \varwrt{\pi}{h}\tau := \varwrt{\pi}{h} \sum_{k \in \mathrm{Z}} \text{Corr}\left(h(X_0), h(X_k)\right)$.\footnote{In the original paper by \citet{roberts2004} the statement of this theorem differs in that they write $\tau = \sum_{k \in \mathrm{Z}} \text{Corr}\left(X_0, X_k\right)$.
  We believe that this is an error as \citet{haggstrom2007} state in their comparison of different ways of writing the asymptotic variance that $\sigma^2(h) = \sum_{k \in \mathrm{Z}} \Cov{h(X_0), h(X_k)}$.
  Now if we use that $X_0 \sim \pi$ we get $\sigma^2(h) = \sum_{k \in \mathrm{Z}} \Cov{h(X_0), h(X_k)}= \var{h(X_0)} + \sum_{k \neq 0} \Cov{h(X_0), h(X_k)} = \varwrt{\pi}{h}(1 + \sum_{k \neq 0} \Cov{h(X_0), h(X_k)}) / \var{h(X_0)} =\varwrt{\pi}{h}(1 + \sum_{k \neq 0} \text{Corr}{(h(X_0), h(X_k))}) = \varwrt{\pi}{h} \sum_{k\in\mathrm{Z}} \text{Corr}{(h(X_0), h(X_k))}$.}
  Assume the Markov chain is $\phi$-irreducible, aperiodic, reversible with respect to $\pi$ and that $\sigma^2(h) < \infty$. Then $\pi$ is the stationary for the chain and
  \begin{align}
    \sqrt{n}\left(\frac{1}{n}\sum_{i=1}^n h(X_i) - \Expwrt{\pi}{h} \right) \overset{d}{\longrightarrow} \normal{0, \sigma^2(h)} \,.
  \end{align}
\end{theorem}
\begin{proof}
  See \citet{roberts2004} Proposition 1 for the first claim and Theorem 27 for the second; see \citet{kipnis1986} for a complete proof of the second claim.
\end{proof}

\noindent
We end this paragraph by discussing the assumptions of Theorem \ref{thm:mcclt} on an intuitive level.

\textbf{\emph{$\phi$-irreducibility}} assumes that we can find a measure $\phi$ so that no matter where the chain starts, we eventually reach every region of the state space which has positive measure with respect to $\phi$.
In the next paragraph we will see that we can make sure that this condition is satisfied by construction such that the chain is $\pi$-irreducible.

\textbf{\emph{Aperiodicity}} assumes that we cannot find disjoint regions on which the chain jumps from one region to another in a cyclical predictable fashion.
It seems intuitive that such a behavior will prevent the chain of actually converging to its stationary distribution.

\textbf{\emph{Reversibility with respect to $\pi$}} is a technical assumption which is best explained by its implications.
In particular, it implies that the Markov chain has $\pi$ as its stationary distribution (which is unique by the other assumption).
Again, later we will see that we can construct a chain which fulfills this condition with $\pi$ being equal to the posterior distribution.

At last, how do we interpret the \textbf{\emph{finite variance}} assumption?
As we assume square integrability of $h$ we get that $\sigma^2(h)$ is finite if and only if the integrated correlation time $\tau$ is finite.
This happens if $\text{Corr}(h(X_0), h(X_k))$ goes fast enough to zero.
We then get the usual large sample variance approximation of the unnormalized sample mean: $\sigma^2(h) / n = \varwrt{\pi}{h} / ({}^n/{}_\tau)$.
In this sense we might say that ${}^n/{}_\tau$ denotes the \emph{effective sample size}, which corrects for the fact that we are not drawing independent samples and therefore need more samples to yield the same amount of information as in the independent case.

\paragraph{Metropolis-Hastings Algorithm.}
From the above we know that given a Markov chain with invariant distribution equal to the posterior distribution $p(\theta \mid z)$, we can treat the realizations of the chain as samples from the posterior, under some regularity conditions.
Here we consider one method which implicitly defines such a chain, namely the Metropolis-Hastings algorithm (\citet{Metropolis1953}, \citet{hastings70}).
For other approaches and more involved algorithms see for example \citet{roberts2004} or \citet{liang10}.

\begin{algorithm}
\caption{Metropolis-Hastings}\label{alg:metropolis-hastings}
\begin{algorithmic}[1]
  \Require $(\pi, q, T) =$ (target density, proposal density, number of samples to draw)
\State initialize $x_0$ with an arbitrary point from the support of $q$
\For{$t = 0,\mydots,T-1$}
  \State sample a candidate: $y \sim q(\cdot \mid x_t)$
  \State compute the acceptance probability: $\mathcal{A} \gets \min\left\{ \ddfrac{\pi(y)}{\pi(x_t)} \ddfrac{q(y \mid x_t)}{q(x_t \mid y)}, 1\right\}$
  \State update the chain: $x_{t+1} \gets \begin{cases} y &\mbox{,with probability } \mathcal{A}\\ x_t &\mbox{,with remaining probability} \end{cases}$
\EndFor{}
\State \textbf{return} $\{x_t : t = 1,\mydots,T\}$
\end{algorithmic}
\end{algorithm}

\noindent
Algorithm \ref{alg:metropolis-hastings} displays the Metropolis-Hastings algorithm.
Line 4 shows why we can use this algorithm with the unnormalized posterior, as the integration constant cancels out in the first fraction.
For different settings different proposal distributions are appropriate.
A common choice are so called \emph{random walk} proposals which add some random number to the current position of the chain; for example a gaussian random walk proposal is given by $q(\cdot \mid x_t) = \normal{\cdot \mid x_t, \sigma^2}$ or equivalently stated $y = x_t + \normal{0, \sigma^2}$.
If the resulting chain has an unique invariant distribution we only know that after some time the chain starts behaving accordingly.
Therefore in practice we choose $T = B + T^*$ and drop the first $B$ samples, where $B$, the so called \emph{burn-in} samples, is large and $T^*$ represents the actual size of samples we want to draw.
See \citet{sherlock2010} for a recent survey on random walk proposals.

The simplicity of the algorithm is remarkable, but the main question of concern is if the resulting Markov chain inherits favorable properties.
And indeed this is the case.
The algorithm creates by construction Markov chains which are reversible with respect to $\pi$ and aperiodic, and if additionally the proposal density is positiv and continuous and $\pi$ is finite then the chain is $\pi$-irreducible; see for example \citet{roberts2004}.

The ability to sample draws in complex settings using the Metropolis-Hastings algorithm (and other Markov chain Monte Carlo methods for that matter) made Bayesian statistics applicable for real problems.
Still, \citet{Au2001EstimationOS} show that the classical Metropolis-Hastings algorithm is highly dependent on the proposal density and fails in higher dimensions; \citet{zuev08} provide a geometric intution.
The following paragraph illustrates one of the problems of working in higher dimensions.

\paragraph{Volume in Higher Dimensions}
Classical MCMC methods can have too slow convergence rates; In higher dimensions this might be due to probability mass being distributed very far from where it is expected (\citet{betancourt2017convergence}).
In this paragraph we motivate this phenomenon and in the following we present methods which utilize it.

Let $B_d$ denote the unit ball in $\reals^d$ and define $C_d$ as the smallest cube containing $B_d$.
We consider two questions.
First, how does the ratio $\nicefrac{\vol{B_d}}{\vol{C_d}}$ change as $d$ increases.
And second, how does the ratio of probability mass distributed by a standard gaussian on these regions change as $d$ increases.
Since closed form expressions of volumina of geometrical objects exist the first questions needs little work.
Similary we can easily compute $P(X \in C_d) = \left[\Phi(1) - \Phi(-1)\right]^d$, where $\Phi$ denotes the one-dimensional gaussian cumulative distribution function.
However, to compute $P(X \in B_d)$ we need to integrate over the unit ball with respect to a gaussian distribution, which is non-trivial.
For this reason we decide to report an upper bound, as this is sufficient for our motivation.
In particular we compute $\overline{P(X \in B_d)} := \sup_{\bm{x} \in B_d} \phi(\bm{x}) \vol{B_d} = \sup_{\bm{x} \in B_d} \phi(\bm{x}) \int_{B_d}1 \mathrm{d}\bm{x} \geqslant \int_{B_d} \phi(\bm{x}) \mathrm{d}\bm{x} = P(X \in B_d)$.
The results of these computations are depicted in table \ref{tab:vol_high_dim}.
We note that both ratios tend to zero very fast as $d$ increases.
With this phenomenon in mind one has to be cautious when working in high-dimensional spaces, since the regions of interest, that is the regions containing non-negligible probability mass, might not be located where our low-dimensional intution says.
This idea is formalized by the \emph{Gaussian Annulus Theorem} (\citet{blum2017foundations}; theorem 2.9) which states, inter alia, that most probability mass lies within an annulus centered at the origin with an average distance to the origin of $\sqrt{d}$.

\begin{table}[ht]
\def\arraystretch{1.3}
\centering
 \begin{tabular}{c | c c c c c c c}
 $d$ & 1 & 2 & 3 & 5 & 7 & 10 & 15\\
 \hline
 ${\vol{B_d}} / {\vol{C_d}}$ & 1.00000 & 0.78540 & 0.52360 & 0.16449 & 0.03691 & 0.00249 & 0.00001\\
 ${\overline{P(X \in B_d)}} / {P(X \in C_d)}$ & 1.16874 & 1.07281 & 0.83589 & 0.35870 & 0.10995 & 0.01184 & 0.00012
 \end{tabular}
 \caption{Comparison of volume ratio of unit ball and cube, and probability ratio of gaussian falling in unit ball and cube for varying dimension $d$. Numbers are rounded to five decimal places.}
\label{tab:vol_high_dim}
\end{table}

We have seen some unintuitive behavior in higher dimensions which might explain why regular methods do not work or only work very slowly.
The next paragraph presents one method which utilizes this behavior to efficiently produce samples.

\paragraph{Hamiltonian Monte Carlo.}
We conclude our digression on Bayesian thinking by presenting \emph{Hamiltonian Monte Carlo} (HMC), an innovative Markov chain Monte Carlo method from the statistical physics literature which works in higher dimensions  \citet{duane87}.
There are of course multiple MCMC algorithms which work in higher dimensions with many more being actively developed.
Here we focus on HMC as it is the main algorithm used in the probabilistic programming language STAN (\citet{standev2018stancore}), which we will be using in our Monte Carlo study and application part.

We note that it is impossible to provide a rigorous introduction to HMC here, which is why we will focus on the general intution and refer to a series of papers by Michael Betancourt and several coauthors on, the geometric foundations of HMC (\citet{betancourt2014geometric}); geometric ergodicity of HMC (\citet{livingstone2016geometric}) and HMC for hiearchical models (\citet{betancourt2013hamiltonian}).\footnote{Besides doing theoretical research on HMC, Michael Betancourt worked for STAN on integrating the HMC algorithm and runs an educational blog where he presents his research using modern tools, see \url{https://betanalpha.github.io/}.}

One reason why ordinary MCMC methods might converge only very slowly in higher dimensions is that the proposal distribution used in the Metropolis-Hastings algorithm does not properly capture the geometry of the high-dimensional space which leads to many rejected proposals and therefore an inefficient exploration of the parameter space.
The main idea of HMC is to extend the parameter space by constructing a specific vector field on it, which moves the chain from one proposed point to another in a way so that we consider points that lie in regions with high probability mass and we regulary jump to faw away points as to explore the space as quickly as possible.
But how do we construct this vector field?
Note that when considering differentiable posterior densities the gradient defines a vector field.
However, this vector field points to the modes of the posterior and as we saw in the last paragraph, in higher dimensions we will find little to no probability mass near the modes.
This is where Hamiltonian mechanics comes into play by providing a set of equations (Hamilton's equations) that describe the time-evolution of the interplay of kinetic and potential energy of a system.
What this means for our case is best explained by imagining the mode as the center of gravity, with gravity pulling harder as we get closer to the mode ---this can be thought of as the gradient vector field.
But as most probability mass is spread around an annulus around the mode we do not want to move closer to the center of gravity, we want to move around an orbit around the mode.
The distance of the orbit to the mode and exact shape depent of course on the dimensionality and posterior distribution of the problem.
Hamilton's equations provide us with a way to construct a vector field so that moving along the field drifts the Markov chain into this orbit.
Once reached the chain explores the relevant space quickly.

Why is the above important?
Bayesian statistics and its application to real world problems has gained immense popularity with the invention of Markov chain Monte Carlo methods.
It's naive application to general complex high-dimensional problems is not computationally feasible however.
These problems are under active development and new approaches on the algorithmic and theoretical side, as the one presented above, prove fruitfull.

\section{Hierarchical Models}
In the following subsections we will introduce the general idea of Hierarchical
models; show how to solve for the posterior analytically in a simplified setting
and then present the main model with which we will be working later on, the \emph{Hierarchical Linear model}.

\subsection{Hierarchical Data}

\subsection{Solving for the posterior analytically}
As in the previous section, we first will be presenting the analytical
derivation of the posterior in a simple normal hierarchical model.

\subsection{Hierarchical Linear Models}

\subsubsection{Varying Slopes Model With One Predictor In Each Level}
We assume that units $i = 1,\mydots,n$ can be divided into $J$ distinct groups.
We start with a very simple model assuming that intercept is fixed for all
groups, that is
\begin{align}
  y = \alpha + \beta_j x + \epsilon \,,
\end{align}
with $\epsilon$ following a mean zero normal distribution with variance
$\sigma_{\epsilon}^2$.
To incorporate the idea that the groups follow a common structure we also
assume
\begin{align}
  \beta_j &= \gamma_0 + \gamma_1 u_j + \eta \,,
\end{align}
for $j = 1,\mydots,J$, with $\eta$ mean zero normal with variance $\sigma_\eta^2$.

Since $\gamma_0$ and $\gamma_1$ do not vary by group they are sometimes referred
to as \emph{fixed effects}. Similary as $\eta$ is drawn randomly for each group
it is sometimes called \emph{random effect}. Put together this shows the close
resemblance of the hierarchical linear model to classical mixed effects models
\textcolor{red}{(some reference here would be nice!)}

Following the notation of \textcolor{red}{Gelman and Hill (2007)} we describe the model equation of a single individual $i$ by
\begin{align}
  y_i = \alpha + \beta_{j[i]} x_i + \epsilon_i \,,
\end{align}
where $j[i]$ denotes the group to which individual $i$ belongs.

\subsubsection{Varying Intercept and Slope Model with One Predictor in Each Level}
We assume that units $i = 1,\mydots,n$ can be divided into $J$ distinct groups.
In each group $j$ we model our outcome variable $y$ as a linear function in $x$,
that is
\begin{align}
  y = \alpha_j + \beta_j x + \epsilon \,,
\end{align}
with $\epsilon$ following a mean zero normal distribution with variance
$\sigma_{\epsilon}^2$.
To incorporate the idea that the groups follow a common structure we also
assume
\begin{align}
  \alpha_j &= \gamma_0^{\alpha} + \gamma_1^{\alpha} u_j + \eta_\alpha \\
  \beta_j &= \gamma_0^{\beta} + \gamma_1^{\beta} u_j + \eta_\beta \,,
\end{align}
for $j = 1,\mydots,J$, with
\begin{align}
  \sqmat{\eta_\alpha \\ \eta_\beta} \sim \normal{\sqmat{\sigma_\alpha^2 & \rho \sigma_\alpha \sigma_\beta \\ & \sigma_\beta^2}}
\end{align}

Following the notation of \textcolor{red}{Gelman and Hill (2007)} we describe the model equation of a single individual $i$ by
\begin{align}
  y_i = \alpha_{j[i]} + \beta_{j[i]} x_i + \epsilon_i \,,
\end{align}
where $j[i]$ denotes the group to which individual $i$ belongs.
This particular model is known as the two-level varying intercept / varying slope model with one unit-level predictor (here $x_i$) and one group-level predictor
(here $u_i$). The model defined by equations \textcolor{red}{(1) - (4)} can of
course be made arbitrarily complex by adding higher order polynomial terms or
more predictors, as in a regular linear regression models. Further the normality
assumption of equation \textcolor{red}{(4)} is not mandotory and can be swapped
with nearly any other distributional assumption. Also, why stop at two levels?
We could naturally model the coefficients in equation \textcolor{red}{2 and 3}
using a third level. All these extensions can in practice be necessary when
modelling complex structures; however for the sake of simplicity and clarity
we will stick to our clean model.

\section{Monte Carlo Study}

\subsection{Convergence}
\begin{enumerate}
 \item Practioners face mutliple problems when trying to apply Bayesian models. A prominent example is the selection of a right prior.

\item The other important consideration is checking convergence of the mcmc chain. Asymptotic theory tells us that the MCMC will converge with a probability of one to the true density for an unlimited number of steps. Practioners are interested in the performance after only a limited number of steps. 
Typically we initate our chain with a number of steps we discard later (burn-in) and test convergence based on the rest of the draws.
\item  
The easiest approach to check convergence is a mere graphical analysis. If the MCMC reached the underlying distribution, new parameters should be drawn around the the mean of the modell. Therefore, the timeseries of the draws should look similar to a stationary process. If the underlying distribution is not reached yet, a slope should be observed. 

\item We can take a more quantitative  approach by calculating a variety of different convergence criterias. Simply spoken, they measure wether different subsection of a chain describe the same underlying distribution. One of the simplest approaches is based on Geweke(1992) and compares the mean of the draws in one subsection of the chain to an other. Inutitively, both should be the same. One diffulty lies in the correction of means by standard deviations, which need to be adjusted for the autocorrelation as draws are not independent from each other. The underlying test is a t-test $\mathbf{E}\left[g(\theta) \mid Y^T\right], i \in {A,C}$
$$\mathbf{CD}_{GW K}=\hat{G}_{S_A}$$


\item Our initial parameter values might habe a sizable effect on our reached distribution. That is why another part of the literature (based on Brooks and Gelman) focuses on starting with different values and comparing the effects on final posterior. If the parameters estimation of the multiple chains align, we can be more convinced that we hit the true distribution of the chain. 

%\item the variance within a chain is
%$$\sigma_j^2=\frac{1}{ N-1} \sum_{n=1}^{N} (g_{nj}-\overline{g_j} )^2 $$
% changed notatiation to follow stan reference manual
%$$W=\frac{1}{ J(N-1)} \sum_{j=1}^{J} \sum_{n=1}^{N} (\overline{g_{nj}}-\overline{g_j} )^2=\frac{1}{J} \sum_{j=1}^{J}\sigma_j ^2 $$


\item the variance between sequence variance B/N is given by
$$B=\frac{N}{M-1} \sum_{m=1}^{M} (\overline{\theta}_m^{(\bullet)}-\overline{\theta}_{\bullet}^{(\bullet)} )^2 $$

\item where 
$$\overline{\theta}_m^{(\bullet)}=  \sum_{n=1}^{N} \theta_m^{(n)}$$

\item and
$$\overline{\theta}_{\bullet}^{(\bullet)}=\frac{1}{M} \sum_{m=1}^{M}  \theta_m^{(\bullet)}$$


\item The within-chain variance is averaged over the chains,

$$W=\frac{1}{M} \sum_{m=1}^{M} s_m^2$$

\item where 
$$s_m^2=\frac{1}{N-1} \sum_{m=1}^{M} (\theta_m^{(n)}-\overline{\theta}_{m}^{(\bullet)} )^2$$

\item and
$$\overline{\theta}_{\bullet}^{(\bullet)}=\frac{1}{M} \sum_{m=M}^{N}  \theta_m^{(\bullet)}$$


\item The variance estimator is a mixture of the within-chain and cross-chain sample variances,
$$\widehat{var}^+ (\theta \mid y)=\frac{N-1}{N}W+\frac{1}{N}B$$


\item Finally, the potential scale reduction statistic is defined by the equation,
$$\widehat{R}=\frac{\widehat{var}^+ (\theta \mid y)}{W}$$

\item If the Markov Chain is converged$ \widehat{R}$ should be close to 1. Intuitively the variance within a chain should create all the variation of the draws, while the variance between different chains converges to 0.



\subsection{Prior selection}
\item The selection of a right prior is one critical part of bayesian modelling. And the often the subject to critizism.

\item Following Gelman, we can differentiate between 5 types of priors
 
\item fllat prior
\item Super-vague but proper prior: normal(0, 1e6);
\item Weakly informative prior, very weak: normal(0, 10);
\item Generic weakly informative prior: normal(0, 1);
\item Specific informative prior: normal(0.4, 0.2) or whatever. Sometimes this can be expressed as a scaling followed by a generic prior: $theta = 0.4 + 0.2*z; z ~ normal(0, 1)$

\item The flat pior (often called uniformative prior) . Consequently, the posterior collapses to the Maximum Likelihood. 
(from https://github.com/stan-dev/stan/wiki/Prior-Choice-Recommendations)
\item Another option is a super




\subsection{Technical considerations}

\item some stuff about bad or good mixing



\subsection{Stan}

\item For our Bayesian Analysis we use the software Python as well as the software Stan through the interface Pystan
\item Stan compiles the code directly into C and therefore allows the fast analysis need for our monte carlo study.

\item Stan allows a great amount of  parametrization. For simplicity we will only focus on a small number of options
\item delta is the metropolis acceptance rate. As shown in above section, mcmc lead to autocorrelated draws. 
We can therfore set an acceptance rate $delta \in[0,1]. With this probability we accept a new draws with a lower posterior value. Why?

\item 
A too high acceptance rate will lead to too many draws to be accepted and the chain to wander widly around.
As a result the autocorrelation we have a high autocorrelation between each draws.

\item 
A too low acceptance rate will lead to only values in the middle of the posterior to be accepted. We have a only slowly decaying autocorellation function again.

\item this can be analyzed looking at  the autocorrelation plot.(insert some plots here with good or bad mixing)

\item A delta of 0.8 is default. (We change this based on our parametrization)
 
\item we vary J and N and check the performance of our Bayesian Estimation with the true results 


\subsection{convergence tests}

\item by not setting any starting values stan start automatically with
\item diffuse random initializations automatically satisfying the declared parameter constraints.
% https://mc-stan.org/docs/2_20/reference-manual/notation-for-samples-chains-and-draws.html


\subsection results

\item We cocentrate on 3 different cases in our Simulation study: flat prior, informative true prior and informative wrong prior. 
\item we vary J and N and check the performance of our Bayesian Estimation with the true results 


\item Based on the package stan-utilty we perform 2 peformance tests in our bayesian analysis
\item First we test wether the empirical percentiles are similar to  




\end{enumerate}

\section{Application}
\label{sec:application}

In this section we discuss and review literature on the application of Bayesian hierarchical modelling approaches. We then use microdata on student testscore perfomance from the UK to give a practical illustration of the work we have developed in the previous sections of our paper and compare that to the frequentist approach to analyzing multilevel data with the aim of showcasing what differences/similarities there are between the two methods. We then depart from the comparative analysis to look deeper into the Bayesian approach and conduct robustness checks that vary certain aspects using Bayesian sensitivity analysis. We close the section off by a brief discussion of the limitations and challenges encountered in this section as an outlook for future work.


\subsection{Literature on the application of Hierarchical Models}

Bayesian hierarchical models or multilevel models are a suitable approach to consider social contexts as well as individual respondents or subjects. It becomes attractive to consider hierarchical models in place of the common (or popularized) frequentist approach as soon as there is a need to relax the independence of residuals assumption as a result of similarities in the characteristics of a group of respondents or when the researcher seeks to disentangle variability at various levels of the data. These models have been used in various applications throughout fields in research. In education research, \cite{buric2020teacher} use these models to examine the relationship between teacher self-efficacy (TSE), instructional quality (i.e., classroom management, cognitive activation, and supportive climate) and student motivational beliefs (i.e., self efficacy and intrinsic motivation) by using responses from both teachers and students and implementing a sophisticated doubly latent multilevel structural equation modelling approach. The results reflect the necessity to disentangle variability at various levels of the data as the researchers find that, at class level, TSE was positively related to the three dimensions of instructional quality but not to students' motivational beliefs. They also find, as expected, that instructional quality was positively related to students’ motivational beliefs.

In agricultural research, \cite{ ramsey2019saying} employ bayesian methods to estimate a two-level model with farm households nested within towns to be able to make inferences on the different farm household behaviours across cities. They estimate the effect of off-farm opportunities on farm exit rates.  Off-farm employment opportunities are thought to influence farm exit rates, though evidence on the sign of this effect has been mixed. Examining this issue in the context of Japanese agriculture, the researchers find that off-farm exit rates are determined by the nature of the off-farm work available to the operators in a town or in which the
operators are already engaged. Towns with higher shares of farm households with no farm sales have higher rates of exit. Also, high shares of off-farm work with income exceeded by the farm income are associated with fewer exits from farming. Bayesian methods allow then to formulate policy reccomendations based on the different behaviours of farming households.

In development economics, \cite{ meager2019understanding} jointly estimates the average effect and the heterogeneity in effects across seven studies using Bayesian hierarchical models to answer questions about external validity that impede consensus on the results from randomized evaluations of microcredit. The researcher uses hierarchical models to disentangle sampling variation from the variation oberved across studies. Hierarchical models address this by jointly modeling sampling variation and true heterogeneity across studies on parallel randomized experiments.
The researcher finds reasonable external validity: true heterogeneity in effects is moderate, and approximately 60 percent of observed heterogeneity is sampling variation. This paper has the potential to revolutionize the field of development economics as the researcher provides a method to establish external validity using multiple studies from different countries. 

On this backdrop, we now demonstrate below, an application to education microdata of a comparison between the likelihood (frequentist)  and bayesian inference approaches. For each of these approaches, we fit a basic varying intercept and slope multilevel linear model with one predictor. We will use the \textit{lmer} function in the \textit{lme4} package for R to determine maximum likelihood estimates of the parameters in linear mixed-effects models. For the Bayesian approach, we then the use \textit{rstanarm} package in R as well. We seek to draw comparisons between the results of both estimation techniques. We expect there to be a difference in the random effects residuals as \cite{browne2006comparison} argue that the Maximum Likelihood approach does not take into account all the uncertainty in a multilevle model.


\subsection{Frequentist and Bayesian Approaches in Practice: Application to Education Data}

A common feature of data structures in education is that units of analysis (e.g., students) are nested in higher organizational clusters (e.g. schools). This kind of structure induces dependence among the responses observed for units within the same cluster. Students in the same school tend to be more alike in their academic and attitudinal characteristics than students chosen at random from the population at large. We will see in this section, a comparison between frequentist and Bayesian approaches, and why a researcher should consider opting for Bayesian hierarchical models in the case of multilevelled data.

\subsubsection{The data}
We will be analyzing the GCSE dataset from \cite{rasbash2000user}. The data include General Certificate of Secondary Education (GCSE) exam scores of 1,905 students from 73 schools in England on a science subject. The GCSE dataset consists of the following 5 variables:
\begin{itemize}
	\item \textit{school}: school identifier
	\item \textit{student}: student identifier
	\item \textit{gender}: gender of a student (M: Male, F: Female)
	\item \textit{written}: total score on written paper
	\item \textit{course}: total score on coursework paper
\end{itemize}
Two components of the exam were recorded as outcome variables: written paper and course work. In this application, only the total score on the coursework paper (course) will be analyzed. In our example, the model estimates the effect of gender on test scores.

\subsubsection{Likelihood inference approach}
In this subsection, we fit a basic varying intercept, varying slope multilevel linear model with one predictor using the \textit{lmer()} functions. Functions such as lmer are based on a combination of maximum likelihood (ML) estimation methods of the model parameters, and empirical Bayes (EB) predictions of the varying intercepts and/or slopes resulting in the Best Linear Unbiased Predictions (BLUPs) of the model parameters. We use these functions so that our parameter estimates from both the ML and the Bayesian framework are comparable.

\subsubsection*{Model: Varying intercept, varying slope with a single predictor}
\label{subsubsection:model}
In this model, we allow both the intercept and the slope to vary randomly across schools using the following model:

\begin{align}
	y_{ij} &= \alpha_j + \beta_j x_{ij} +\epsilon_{ij} \,, \text{ where } \epsilon_{ij} \sim \normal{0, \sigma_y^2} \\
	\alpha_j &= \mu_\alpha + u_j \,, \text{ where } u_j \sim \normal{0, \sigma_\alpha^2}  \\
	\beta_j &= \mu_\beta + v_j \,, \text{ where } v_j \sim \normal{0, \sigma_\beta^2}
\end{align}
or in reduced form as $y_{ij} = \mu_\alpha + \mu_\beta x_{ij} + u_j + v_j x_{ij} + \epsilon_{ij}$, where $\epsilon_{ij} \sim \normal{0,\sigma_{y}^{2}}$ and
\begin{align}
	\left( \begin{matrix} u_j \\ v_j \end{matrix} \right) \sim \normal{ \left( \begin{matrix} 0 \\ 0 \end{matrix} \right) ,\left( \begin{matrix} { \sigma  }_{ \alpha  }^{ 2 } & \rho { \sigma  }_{ \alpha  }{ \sigma  }_{ \beta  } \\ \rho { \sigma  }_{ \alpha  }{ \sigma  }_{ \beta  } & { \sigma  }_{ \beta  }^{ 2 } \end{matrix} \right)}.
\end{align}
Under the \textit{Fixed effects} part of the results in Table \ref{tab:results}, we see that the intercept $\mu_{\alpha}$, averaged over the population of schools, is estimated as $69.425$ and the slope $\mu_{\beta}$ as $7.128$. The average regression line across schools is thus estimated as $\hat{y}_{ij} = 69.424 + 7.128 x_{ij}$, with the residual within-school standard deviation estimated as $\hat{\sigma}_{y}=13.03$ (shown under the \textit{Random Effects} part). The estimated standard deviations of the school intercepts and the school slopes are $\hat{\sigma}_{\alpha}= 10.15$ and $\hat{\sigma}_{\beta}=6.92$ respectively. In Table \ref{tab:results} under the \textit{Random effects} part, we do not report the estimates as they are always zero for the within and between school variances.

\begin{table}[H]
	\centering
	\caption{{\small Comparison of Maximum Likelihood and Bayesian estimates.}}
	\label{}
	
	\smallskip
	\begin{tabular}{l*{2}{c}}
		\toprule \\[-1.0em]
		\multicolumn{3}{c}{Dependent variable: Course test score}\\ \\[-1.0em]
		&ML &Bayes\\
		\midrule \\[-1.0em]
		\emph{A. Random effects}\\
		Intercept & -- & --\\
		& (10.146) & (10.249)\\
		Female & -- & --\\
		& (6.924) & (7.099)\\
		Residual & -- & --\\
		& (13.030) & (13.0)\\
		\emph{B. Fixed effects} \\
		Intercept & 69.425 & 69.413\\
		& (1.352) & (1.287)\\
		Female & 7.128 & 7.132\\
		& (1.131) & (1.165)\\
		\hline
		Students&1725&1725\\
		Schools&73&73\\
		\bottomrule
		
	\end{tabular}
	\label{tab:results}
\end{table}
Treating the estimates of $\mu_\alpha$, $\mu_\beta$, $\sigma^2_{y}$, and $\sigma^2_{\alpha}$ as the true parameter values, we can then obtain the Best Linear Unbiased Predictions (BLUPs) for the school-level residuals $\hat{u}_j = \hat{\alpha}_{j} - \hat{\mu}_{\alpha}$ and $\hat{v}_j = \hat{\beta}_{j} - \hat{\mu}_{\beta}$. The BLUPs are equivalent to the so-called Empirical Bayes (EB) prediction, which is the mean of the posterior distribution of $u_{j}$ and $v_{j}$ given all the estimated parameters, as well as the random variables $y_{ij}$ and $x_{ij}$ for the cluster.  These predictions are called "Bayes" because they make use of the pre-specified prior distribution \footnote{We elaborate more on prior distributions in Section the full Bayesian approach section} $\alpha_j \sim \normal{\mu_\alpha, \sigma^2_\alpha}$, and by extension $u_j \sim \normal{0, \sigma^2_\alpha}$ (analogous for the slope $\beta_j$), and called "Empirical" because the parameters of these priors, $\mu_\alpha$, $\mu_\beta$,  $\sigma^2_{\beta}$ and $\sigma^2_{\alpha}$, in addition to $\sigma^2_{y}$, are estimated from the data.

Compared to the Maximum Likelihood (ML) approach of predicting values for $u_j$ and $v_j$ by using only the estimated parameters and data from cluster $j$, the EB approach additionally considers the prior distributions of $u_{j}$ and $v_{j}$, and produces predicted values closer to $0$ (a phenomenon described as shrinkage or partial pooling).  To see why this phenomenon is called shrinkage, we can take an example of the intercept $u_j$ and express the estimates obtained from EB prediction as $\hat{u}_j^{\text{EB}} = \hat{R}_j\hat{u}_j^{\text{ML}}$ where $\hat{u}_j^{\text{ML}}$ are the ML estimates, and $\hat{R}_j = \frac{\sigma_\alpha^2}{\sigma_\alpha^2 + \frac{\sigma_y^2}{n_j}}$ is the so-called Shrinkage factor. This would be analogous for $v_j$ as well.

By using the \textit{ranef} function, we can retrieve information on how much the intercept and slope are shrunk up or down in particular schools. For example, in the third school (22710) in the dataset shown in Table \ref{tab:shrinkage} (left), the estimated intercept is $10.66$ higher than average and the estimated slope is $-4.31$, so that the school-specific regression line is $(69.43 + 10.66) + (7.13 - 4.31) x_{ij}$ which turns out to be $ 80.09 + 2.82 x_{ij}$. This suggests that although the fixed effect average is positive, (thus on average and across schools, female students perform better than males), in this particular school, female students do on average perform $-4.31$ points less than the across-school average female test score.

\begin{table}[!htb]
	\caption{Global caption}
	
	\begin{minipage}{.5\linewidth}
		\caption{}
		\centering
		{
			\begin{tabular}{l | c c c c c}
				School ID & Intercept & Slope (FemaleF)\\
				\hline
				20920 & -19.25 & 14.64 \\
				22520 & -20.18 & 4.73 \\
				22710 & 10.66 & -4.31 \\
				22738  & 0.63 & -0.029 \\
				22908 & -3.46 & -6.13 \\
				23208 & 7.85 & -4.41
			\end{tabular}
		}
	\end{minipage}%
	\begin{minipage}{.5\linewidth}
		\centering
		\caption{}
		{
			\begin{tabular}{l | c c c c c}
				School ID & Intercept & Slope (FemaleF)\\
				\hline
				20920 & -18.79 & 14.16 \\
				22520 & -20.07 & 4.60 \\
				22710  & 10.46 & -3.90 \\
				22738  & 0.73 & -0.083  \\
				22908 & -3.62 & -5.97 \\
				23208  & 7.77 & -4.33
			\end{tabular}
		}
		
	\end{minipage}
	\caption{{\small Shrinkage factor for the ML estimates (left) and Bayesian estimates (right): an extract of 6 schools}}
	\label{tab:shrinkage}
\end{table}

We can also retrieve the information on how much the intercept and slope are shrinking up or down for the full bayesian estimation method to compare them to the BLUPs as shown in Table \ref{tab:shrinkage} (right). We indeed find that, for an extract of the first six schools, there is not a significant difference in the estimates except for a few instances. The BLUPs then allow us to compare results between ML and Bayes estimates.

\subsubsection*{Partial Pooling in multilevel models}
\cite{gelman2006data} characterize multilevel modelling as partial pooling (also called shrinkage), which is a compromise between two extremes: complete pooling in which the clustering is not considered in the model at all, and no pooling, in which separate intercepts are estimated for each school as coefficients of dummy variables. The estimated school-specific regression lines in the above ML model (obtained from BLUPs as shown above) are based on partial pooling estimates. To show this, we first estimate the intercept (the same could be done for the slope) in each school in three ways; 1) complete pooling with OLS, 2) no pooling with OLS, and 3) partial pooling with ML. We then plot the data and school-specific regression lines for a selection of eight schools as illustrated in Figure \ref{fig:pooling}. We see that the estimated school-specific regression line from the partial pooling estimates lies between the complete-pooling and no-pooling regression lines. There is more pooling (purple dotted line closer to red dotted line) in schools with larger sample sizes. By using the BLUPs method we ensure that the ML estimates are partial pooling estimates and therefore are comparable to the Bayesian framework.

\subsubsection{Full Bayesian Inference Approach}
As previously mentioned, functions such as \textit{lmer} are based on a combination of maximum likelihood (ML) estimation of the model parameters, and empirical Bayes (EB) predictions of the varying intercepts and/or slopes. However, in some instances, when the number of groups is small or when the model contains many varying coefficients or non-nested components, the ML approach may not work as well in part because there may not be enough information to estimate variance parameters precisely. In such cases, a fully Bayesian approach provides reasonable inferences with the added benefit of accounting for all the uncertainty in the parameter estimates when predicting the varying intercepts and slopes, and their associated uncertainty. This is one of the reasons why a fully Bayesian estimation is particularly interesting. Other reasons are discussed in section \ref{subsection:Deeper}. We now demonstrate below, how to fit the model from section \ref{subsubsection:model} in a fully Bayesian framework using the \textit{rstammarm} package. This package is a wrapper for the \textit{rstan} package that enables the most common applied regression models to be estimated using Markov Chain Monte Carlo (MCMC) but still be specified using customary R modelling syntax.

\subsubsection*{Model: Varying intercept, varying slope with a single predictor}

We can implement a fully Bayesian estimation for multilevel models with only minimal changes to our existing code with \textit{lmer} from the maximum likelihood application in the previous section by prepending \textit{stan\textunderscore} to the \textit{lmer} call. The \textit{stan\textunderscore lmer} function is similar in syntax to \textit{lmer} but rather than performing maximum likelihood estimation, Bayesian estimation is performed via MCMC. As each step in the MCMC estimation approach involves random draws from the parameter space, we include a seed option to ensure that each time the code is run, \textit{stan\textunderscore lmer} outputs the same results. We also need to specify prior distributions for each of our parameters.

\subsubsection*{Specifying prior distributions}

The normal distribution for the $\alpha_{j}$'s and $\beta_{j}$'s can be thought of as a prior distributions for these varying intercepts. In the full Bayesian inference setting, all the hyperparameters ($\mu_{\alpha}$, $\mu_{\beta}$, $\sigma_{\alpha}$ and $\sigma_{\beta}$), along with the other unmodeled parameters (in this case, $\sigma_{y}$) need a prior distribution. For this illustration, we use weakly informative priors that provide moderate regularization and help stabilize computation.

Before accounting for the scale of the variables, $\mu_{\alpha}$ and $\mu_{\beta}$ are given normal prior distributions with mean 0 and standard deviation 10.  That is, for example, $\mu_{\alpha} \sim N(0, 10^2)$. The standard deviation of this prior distribution, 10, is ten times as large as the standard deviation of the response if it were standardized. This should be a close approximation to a noninformative prior over the range supported by the likelihood, which should give inferences similar to those obtained by maximum likelihood methods if similarly weak priors are used for the other parameters. \textit{Rstanarm} scales the priors in relation to the scale of variables in the estimation process. The (unscaled) prior for $\sigma_{y}$ is set to an exponential distribution with the rate parameter set to 1.

Additionally, we are also required to specify a prior for the covariance matrix $\Sigma$ for varying (or "random") effects $\alpha_j$ and $\beta_j$ in this Model.  \textit{Stan\_lmer} decomposes this covariance matrix (up to a factor of $\sigma_y$) into (i) a correlation matrix $R$ and (ii) a matrix of variances $V$, and assigns them separate priors as shown below. 


\begin{align}
	\Sigma &=
	\left(\begin{matrix}
		\sigma_\alpha^2 & \rho\sigma_\alpha \sigma_\beta \\
		\rho\sigma_\alpha\sigma_\beta&\sigma_\beta^2
	\end{matrix} \right)\\ &=
	\sigma_y^2\left(\begin{matrix}
		\sigma_\alpha^2/\sigma_y^2 & \rho\sigma_\alpha \sigma_\beta/\sigma_y^2 \\
		\rho\sigma_\alpha\sigma_\beta/\sigma_y^2 & \sigma_\beta^2/\sigma_y^2
	\end{matrix} \right)\\ &=
	\sigma_y^2\left(\begin{matrix}
		\sigma_\alpha/\sigma_y & 0 \\
		0&\sigma_\beta/\sigma_y
	\end{matrix} \right)
	\left(\begin{matrix}
		1 & \rho\\
		\rho&1
	\end{matrix} \right)
	\left(\begin{matrix}
		\sigma_\alpha/\sigma_y & 0 \\
		0&\sigma_\beta/\sigma_y
	\end{matrix} \right)\\
	&= \sigma_y^2VRV.
\end{align}
The correlation matrix $R$ is 2 by 2 matrix with 1's on the diagonal and $\rho$'s on the off-diagonal. 

\textit{Stan\_lmer} assigns the correlation matrix $R$ an LKJ prior (\cite{lewandowski2009generating}), with regularization parameter 1.  This is equivalent to assigning a uniform prior for $\rho$.  The more the regularization parameter exceeds one, the more peaked the distribution for $\rho$ to take the value 0.

The matrix of (scaled) variances $V$ can first be collapsed into a vector of (scaled) variances, and then decomposed into three parts, $J$, $\tau^2$ and $\pi$ as shown below.

\begin{align}
	\left(\begin{matrix}
		\sigma_\alpha^2/\sigma_y^2 \\
		\sigma_\beta^2/\sigma_y^2
	\end{matrix} \right) =
	2\left(\frac{\sigma_\alpha^2/\sigma_y^2 + \sigma_\beta^2/\sigma_y^2}{2}\right)\left(\begin{matrix}
		\frac{\sigma_\alpha^2/\sigma_y^2}{\sigma_\alpha^2/\sigma_y^2 + \sigma_\beta^2/\sigma_y^2} \\
		\frac{\sigma_\beta^2/\sigma_y^2}{\sigma_\alpha^2/\sigma_y^2 + \sigma_\beta^2/\sigma_y^2}
	\end{matrix} \right)=
	J\tau^2 \pi.
\end{align}



In this formulation, $J$ is the number of varying effects in the model (here, $J=2$), $\tau^2$ can be viewed as an average (scaled) variance across the varying effects $\alpha_j$ and $\beta_j$, and $\pi$ is a non-negative vector that sums to 1 (called a Simplex/probability vector).  A symmetric Dirichlet \footnote{The Dirichlet distribution is a multivariate generalization of the beta distribution with one concentration parameter, which can be interpreted as prior counts of a multinomial random variable (the simplex vector in our context).} distribution with concentration parameter set to 1 is then used as the prior for $\pi$.  By default, this implies a jointly uniform prior over all Simplex vectors of the same size.  A scale-invariant Gamma prior with shape and scale parameters both set to 1 is then assigned for $\tau$.  This is equivalent to assigning as a prior, the exponential distribution with rate parameter set to 1 which is consistent with the prior assigned to $\sigma_y$. (\cite{lewandowski2009generating}) provides a more indepth discussion on the collapsing and decomposition of the matrix $R$.

\subsubsection*{Estimation results}
In Table \ref{tab:results}, we see the point estimate of $\mu_{\alpha}$ from the Bayesian estimation is $69.413$ and this corresponds to the median of the posterior draws.  This is similar to the ML estimate obtained previously ($69.425$).  The point estimate for $\mu_{\beta}$  is slightly different in this Model ($7.132$ compared to the ML estimate $7.128$). When using the Bayesian estimation function, standard errors are obtained by considering the median absolute deviation (MAD) of each draw from the median of those draws.  The Bayesian estimate for $\sigma_{\alpha}$ ($10.249$) and $\sigma_{\beta}$ ($7.099$) are larger than the ML estimate $10.146$ and ($6.924$) respectively. 

\subsubsection*{Remarks on the comparison excercise}
The discrepancy observed between the ML and Bayes residuals may be because the ML approach does not take into account the uncertainty in $\mu_{\alpha}$ and $\mu_{\beta}$ when estimating $\sigma_{\alpha}$ and $\sigma_{\beta}$. Full Bayes, on the other hand, propagates the uncertainty in the hyperparameters throughout all levels of the model and provides more appropriate estimates of uncertainty \cite{browne2006comparison}. Noting this and taking into account the outcome illustrated in the monte carlo study, we conclude that it may be beneficial as a researcher to start by quickly fitting many specifications in building a model using the ML approach, and then taking advantage of the flexibility of a fully Bayesian approach to obtain simulations summarizing uncertainty about coefficients, predictions, and other quantities of interest.

\subsection{Looking Deeper into the Bayesian Approach}
\label{subsection:Deeper}
In education research and practice it is often of interest to compare the schools included in the data. Relevant questions include 1) what are the rankings of these schools within the sample, 2) what is the difference between the means of schools A and B, and 3) is school A performing better than school B? When non-Bayesian methods are used, we can attempt to make such comparisons based on empirical Bayes (or Best Linear Unbiased) predictions of the varying slope and/or intercepts, but it will generally be impossible to express the uncertainty for nonlinear function such as rankings. \cite{goldstein1996league} discuss this in greater detail.

Having samples of all the parameters and varying intercepts and slopes from their joint posterior distribution makes it easy to draw inferences about functions of these parameters.
During estimation, four MCMC chains of 2,000 iterations each are generated. Half of these iterations in each chain are used as warm-up/burn-in (to allow the chain to converge to the posterior distribution), and hence we only use 1,000 samples per chain. These MCMC-generated samples are taken to be drawn from the posterior distributions of the parameters in the model. We can use these samples for predictions, summarizing uncertainty and estimating credible intervals for any function of the parameters.
We can then generate a matrix for varying intercepts $\alpha_j$ and slopes $\beta_j$ as well as vectors containing the draws for the within standard deviations and the between variance by manipulating this matrix.

For the purposes of simplicity and illustration, we will continue our analysis from here onwards using the varying intercept draws. We have saved 4,000 posterior draws (from all four chains) for the varying intercepts $\alpha_{j}$ of the 73 schools. For example, the first column of the 4,000 by 73 matrix is a vector of 4,000 posterior simulation draws for the first school's (School 20920) varying intercept $\alpha_{1}$.  One quantitative way to summarize the posterior probability distribution of these 4,000 estimates for $\alpha_{1}$ is to examine their quantiles by computing mean, SD, median, and $95\%$ credible interval of varying intercepts as shown in Table \ref{tab:summary_data}.

\begin{table}[ht]
	\centering
	\def\arraystretch{1.3}
	{\small
		\begin{tabular}{l | c c c c c}
			School & Posterior mean & Posterior SD & Q2.5 & Q50 & Q97.5\\
			\hline
			b[(Intercept) school:20920] & 50.47 & 5.90 & 38.59 & 50.45 & 61.73 \\
			b[(Intercept) school:22520] & 49.38 & 2.71 & 43.99 & 49.36 & 54.54 \\
			b[(Intercept) school:22710] & 79.84 & 4.36 &  71.42 & 79.87 & 88.54 \\
			b[(Intercept) school:22738] & 70.10 & 4.79 & 60.71 & 70.15 & 79.35  \\
			b[(Intercept) school:22908] & 65.88 & 6.99 & 52.18 & 65.73 & 79.93 \\
			b[(Intercept) school:23208] &  77.15 & 4.52 & 67.96 & 77.24 & 85.87
		\end{tabular}
	}
	\caption{{\small Summary statistics for posterior mean, SD and $95\%$ credible intervals.}}
	\label{tab:summary_data}
\end{table}

\subsubsection{Ranking varying intercepts by school}

We can answer question 1) by producing a caterpillar plot to show the fully Bayes estimates for the school varying intercepts in rank order together with their $95\%$ credible intervals in Figure \ref{fig:ranking}. We can also use the same approach to generate $95\%$ credible intervals for $\beta_j$ $\sigma_y$, $\sigma_\alpha$ and $\sigma_\beta$. From this plot, we are able to identify the schools in which males are performing below or above the mean school perfomance illustrated by the red line.

\subsubsection{Making comparisons between individual schools}

To answer question 2) and 3), we compare two randomly chosen schools as an example: Schools 60501 (the $21^{st}$ school) and 68271 (the $51^{st}$ school). We already have 4,000 posterior simulation draws for both schools. To make inferences regarding the difference between the average scores of the two schools, we can simply take the difference between the two vectors of draws $\alpha_{51} - \alpha_{21}$.

We can investigate the posterior distribution of the difference with descriptive statistics and a histogram. From Figure \ref{fig:differences}, we can see that the expected mean difference is 5.747 with a standard deviation of 6.094 and a wide range of uncertainty. The 95\% credible interval is [-6.55, 17.49], so we are 95\% certain that the true value of the difference between the two schools lies within this range, given the data.

We also can get the proportion of simulation runs that School 60501 has a higher mean than School 68271 as shown in Table \ref{tab:comparison}.

\begin{table}[ht]
	\centering
	\def\arraystretch{1.3}
	{\small
		\begin{tabular}{l | c c c c c}
			FALSE & TRUE \\
			\hline
			17.28 & 82.73
		\end{tabular}
	}
	\caption{{\small The  posterior probability that School 60501 is better than School 68271.}}
	\label{tab:comparison}
\end{table}

This means that the posterior probability that School 60501 is better than School 68271 is 82.7\%. Any pair of schools within the sample of schools can be compared in this manner.

\subsubsection{Convergence}
All chains must converge to the target distribution for inferences to be valid. The diagnostic which we use to assess whether the chains have converged to the posterior distribution is the statistic $\hat{R}$ (\cite{gelman1992inference}). Each parameter has the $\hat{R}$ statistic associated with it and this statistic is automatically generated during estimation.


The $\hat{R}$ is essentially the ratio of between-chain variance to within-chain variance analogous to ANOVA. The $\hat{R}$ statistic should be less than 1.1 if the chains have converged. Figure \ref{fig:rhat} illustrates the $\hat{R}$ statistic and shows that the chains converge.


\subsubsection{Robustness checks}

We apply a concept from Bayesian sensitivity analysis to explore how the inferences change as we change the prior standard deviation. We use three different priors: an informative prior, a weakly informative one and an uninformative prior. Although we vary the means for both $\mu_{\alpha}$ and $\mu_{\beta}$, we report the comparisons in output for $\mu_{\alpha}$ in keeping with the rest of Section \ref{subsection:Deeper}.

\begin{table}[!ht]
	\begin{center}
		\begin{tabular}{l | c c c c c}
			Prior specification & Posterior mean & Posterior SD & Q2.5 & Q50 & Q97.5\\
			\hline
			Uninformative prior $\normal{0,100}$ & 69.44 & 4.37 & 60.85 & 69.44 & 77.98 \\
			Weakly informative prior $\normal{0,10}$ & 69.32 &   4.38  & 60.75 &   69.33 &   77.88 \\
			Informative prior $\normal{0,1}$ & 69.20 & 4.61 &  60.17 &  69.19  & 78.24
		\end{tabular}
	\end{center}
	\caption{The average posterior mean, sd and credible interval monte carlo draws for $\mu_{\alpha}$.}
	\label{tab:robustness}
\end{table} 
As the results in Table \ref{tab:robustness} illustrate, we do not observe a large difference in the estimates which indicates that the Bayesian method is robust to changes in the standard deviation of the class of normally distributed priors. According to \cite{edwards1963bayesian}  the choice of prior within some class of candidate priors has little effect on the posterior, i.e., that the estimates found using one choice of prior within the same class are similar to what is found with another choice in the same distribution class.  

\subsection{Challenges, Limitations and Future Outlook}
For simplicity we have explored an empirical model with varying slopes and intercepts that includes one predictor. In as much as this provides parsimonity in estimation and illustration, there are other features of Bayesian hierarchical models that could be explored beyond our application. For example, by increasing the number of predictors, we can give a more meaningful interpretation to our findings from the estimation. We could also add more features to the school level equations where we vary the slope and intercepts, as illustrated in the monte carlo study. This provides more information to the estimation process about the students in the schools. It would also be interesting to estimate the same model on a non-nested dataset. For future research therefore, we hope to extend this discussion to explore more components of the varying slopes, varying intercept model and linear Bayesian Hierarchical models in general.

\section{Conclusion}



% Markus part

In the simulation study, we demonstrated the effect of various prior distributions on the estimation of hierarchical models. 
We have shown that wrong prior distribution pulls the posterior in its direction. However, the prior also guide the estimation of parameters. A uniform prior does increase the variability in the model estimates. \\

We conclude that it is a good strategy to use weak prior distributions, even when one is uninformed about the actual parameter values. Weak priors will improve the fit of your model when working with small sample sizes, also if they are wrong centered.
Further, we find that the random slope model with group characteristics requires large sample sizes to come up with stable estimates of the group model coefficient. Therefore we use a simple model in the following application of the Bayesian model.
\newpage
\bibliographystyle{chicago}
\bibliography{bibliography}
\appendix
\section{Appendix}

\subsection{Definitions}
\begin{definition}{(Scaled inverse $\chi^2$ distribution).}\label{def:scaledInverseChi}
  Let $\nu > 0$ and $\tau^2 > 0$ be parameters representing degrees of freedom and scale, respectively. The family of \emph{scaled inverse $\chi^2$ distributions} is characterized by its probability density function, namely
  \begin{align*}
    p(x) \propto x^{-(1 + \nu / 2)} \EXP{\frac{-\nu \tau^2}{2 x}} \quad \text{for} \, x \in (0, \infty) \,,
  \end{align*}
  where the constant of integration is ignored for clarity.
  We write $X \sim \scaledInvChi{\nu, \tau^2}$ to denote that the random variable $X$ follows a scaled inverse $\chi^2$ distribution with parameters $\nu$ and $\tau^2$.
\end{definition}

\begin{definition}{(Conjugate prior).}\label{def:conjugate_prior}
Let the likelihood $p(y \mid \theta)$ be given and assume that the prior distribution $p(\theta)$ is a member of some family $\mathcal{F}$ of probability distributions.
We say that $p(\theta)$ is a \emph{conjugate prior} if the posterior $p(\theta \mid y)$ is also a member of $\mathcal{F}$.
\end{definition}

\subsection{Figures}

\begin{figure}[ht]
\begin{center}
\begin{tikzpicture}%
  [vertex/.style={circle,draw=black,fill=white, minimum size=1cm},
  node distance=2.5cm,
  >=latex,
  on grid]
  \node[vertex] (phi) {$\phi$};
  \node[rectangle, draw=black, minimum size=0.9cm,left=2cm of phi] (zeta) {$\zeta$};
  \node[vertex,below left=1.5cm and 2cm of phi] (theta1) {$\theta_1$};
  \node[vertex,below right=1.5cm and 2cm of phi] (thetaJ) {$\theta_J$};
  \node[below=1.5cm of phi] (dots1) {$\dots$};
  \node[rectangle, draw=black, minimum size=0.9cm, left=2cm of theta1] (u1) {$u_1$};
  \node[rectangle, draw=black, minimum size=0.9cm, right=2cm of thetaJ] (uJ) {$u_J$};
  \node[vertex,below=2cm of theta1] (y1) {$y(1)$};
  \node[vertex,below=2cm of thetaJ] (yJ) {$y(J)$};
  \node[rectangle, draw=black, minimum size=0.9cm, left=2cm of y1] (x1) {$x(1)$};
  \node[rectangle, draw=black, minimum size=0.9cm, right=2cm of yJ] (xJ) {$x(J)$};
  \node[below=3.5cm of phi] (dots1) {$\dots$};
  \draw[->]
    (zeta) edge (phi)
    (phi) edge (theta1)
    (phi) edge (thetaJ)
    (u1) edge (theta1)
    (uJ) edge (thetaJ)
    (theta1) edge (y1)
    (thetaJ) edge (yJ)
    (x1) edge (y1)
    (xJ) edge (yJ);
\end{tikzpicture}
\end{center}
\caption{A generic two-level Bayesian hierarchical model depicted as a directed acyclical graph modeling generic observations $y(j)$ in groups $j = 1,\mydots, J$. Circled parameters denote random quantities while parameters contained in squares denote fixed quantities.}
\label{fig:group_sem}
\end{figure}

%\begin{figure}[ht]
%\begin{center}
%\begin{tikzpicture}%
%  [vertex/.style={circle,draw=black,fill=white, minimum size=1cm},
%  node distance=2.5cm,
%  >=latex,
%  on grid]
%  \node[vertex] (phi) {$\mu, \tau^2$};
%  \node[vertex,below left=1.5cm and 2cm of phi] (theta1) {$\theta_1$};
%  \node[vertex,below right=1.5cm and 2cm of phi] (thetaJ) {$\theta_J$};
%  \node[below=1.5cm of phi] (dots1) {$\dots$};
%  \node[vertex,below=2cm of theta1] (y1) {$y(1)$};
%  \node[vertex,below=2cm of thetaJ] (yJ) {$y(J)$};
%  \node[rectangle, draw=black, minimum size=0.8cm, below=2.5cm of phi] (sigma) {$\sigma^2$};
%  \node[below=3.5cm of phi] (dots1) {$\dots$};
%  \draw[->]
%    (phi) edge (theta1)
%    (phi) edge (thetaJ)
%    (theta1) edge (y1)
%    (thetaJ) edge (yJ)
%    (sigma) edge (y1)
%    (sigma) edge (yJ);
%\end{tikzpicture}
%\end{center}
%\label{fig:example_model}
%\caption{The directed acyclical graph representation of the model structure of the example model in subsection \ref{sec:hierachical_solving}. Random  and fixed quantities are depicted in circles and squares, respectively.}
%\end{figure}

\subsection{Tables}

\begin{proof}[Derivation of Results in Table \ref{tab:comp_uniform_bay_ml}.]
  We know that for the problem at hand the standard maximum likelihood estimators
  and their variances are given by
  \begin{align}
    \hat{\mu}_{ML} &= \frac{1}{n} \sum_i y_i = \bar{y}\,\\
    \hat{\sigma}_{ML}^2 &= \frac{1}{n} \sum_i (y_i - \bar{y})^2 = \frac{n-1}{n} s^2\,\\
    \mathrm{I}(\mu, \sigma)^{-1} &= \sqmat{\sigma^2 / n & 0\\0&\sigma^2/(2n)} \,.
  \end{align}
  The Bayesian counterpart to the ML-Estimator is the \emph{maximum a posteriori estimate},
\end{proof}

\begin{proof}[Derivation of Results in Table \ref{tab:comp_conjugate_bay_ml}.]

\end{proof}

\begin{proof}[Derivation of Results in Table \ref{tab:vol_high_dim}]
See file \lstinline{volume.py}{} in the online appendix.
\end{proof}

\subsection{Proofs}
\begin{remark}
The proofs presented here follow \citet{gelmanbda04}; however, we contribute detailed remarks.
\end{remark}

\begin{proof}[Proof of Proposition \ref{prop:posterior_uniform}.]
  Consider first the object $\mu \mid \sigma^2, y$.
  We get
  \begin{align*}
    p(\mu \mid \sigma^2, y) \propto p(y \mid \mu, \sigma^2) p(\mu \mid \sigma^2) \propto p(y \mid \mu, \sigma^2) \,,
  \end{align*}
  where the last step follows as the priors are assumed to be independent.
  Note then
  \begin{align*}
    p(\mu \mid \sigma^2, y) &\propto \EXP{-\frac{1}{\sigma^2}\sum_i (y_i - \mu)^2} = \EXP{-\frac{n}{\sigma^2} \frac{1}{n}\sum_i (y_i^2 - 2y_i \mu + \mu^2)}\\
    &=\EXP{-\frac{n}{\sigma^2} (\bar{y^2} - 2 \bar{y} \mu + \mu^2)} \propto
    \EXP{-\frac{1}{\sigma^2 / n} (\mu - \bar{y})^2} \,,
  \end{align*}
  where $\bar{y^2} = \frac{1}{n}\sum_i y_i^2$ and the last step is only proportional as we switch $\bar{y^2}$ for $\bar{y}^2$.
  Note that proportionality here is with respect to $\mu$.
  We thus get $\mu \mid \sigma^2, y \sim \normal{\bar{y}, \sigma^2/n}$ as our first intermediate result.

  Consider now $\sigma \mid y$.
  As we already derived the joint posterior we can compute the marginal posterior of $\sigma^2$ by integrating out $\mu$.
  Note that $\sum_i (y_i - \mu)^2 = [(n-1)s^2 + n(\bar{y} - \mu)^2]$, where $s^2$ denotes the (unbiased) sample variance. Hence
  \begin{align*}
    p(\sigma^2 \mid y) &\propto \int p(\mu, \sigma^2 \mid y) \mathrm{d} \mu\\
    &\propto \int \sigma^{-(n+2)} \EXP{-\frac{1}{2\sigma^2}\sum_i (y_i - \mu)^2} \mathrm{d} \mu\\
    &= \sigma^{-(n+2)} \int \EXP{-\frac{1}{2\sigma^2}\sum_i (y_i - \mu)^2} \mathrm{d} \mu\\
    &= \sigma^{-(n+2)} \int \EXP{-\frac{1}{2\sigma^2}[(n-1) s^2 + n(\bar{y} - \mu)^2]} \mathrm{d} \mu\\
    &= \sigma^{-(n+2)} \EXP{-\frac{1}{2\sigma^2}[(n-1) s^2]} \int \EXP{\frac{1}{2\sigma^2/n}(\mu - \bar{y})^2} \mathrm{d} \mu\\
    &= \sigma^{-(n+2)} \EXP{-\frac{1}{2\sigma^2}[(n-1) s^2]} \sqrt{2 \pi \sigma^2 / n} \\
    &\propto (\sigma^2)^{-(n+1)/2} \EXP{-\frac{1}{2\sigma^2}[(n-1) s^2]} \,,
  \end{align*}
  where the second to last step follows simply by considering the constant of integration of the normal distribution of $\mu \mid \sigma^2, y$.
  Note that here we consider proportionality with respect to $\sigma^2$.
  By inspection we see that $\sigma^2 \mid y \sim \text{scaled-Inv-} \chi^2(n-1, s^2)$, which proves our first claim.

  To finish the proof we integrate the joint posterior over $\sigma^2$ to get the marginal posterior of $\mu$.
  We evaluate the integral by substitution using
  $z = \sfrac{a}{2 \sigma^2}$ with $a = (n-1)s^2 + n(\mu - \bar{y})^2$.

  Then,
  \begin{align*}
    p(\mu \mid y) &= \int_{(0, \infty)} p(\mu, \sigma^2 \mid y) \mathrm{d}\sigma^2\\
    &\propto  \int_{(0, \infty)} (\sigma^2)^{-(n+2)/2} \EXP{-\frac{1}{2\sigma^2}[(n-1) s^2 + n(\mu - \bar{y})^2]} \mathrm{d} \sigma^2\\
    &\propto \int_{(0, \infty)} (\sigma^2)^{-(n+2)/2}\EXP{-z} [(\sigma^2)^2 / a] \mathrm{d}z\\
    &= \int_{(0, \infty)} (\sigma^2)^{-(n-2)/2} / a \EXP{-z} \mathrm{d}z\\
    &= a^{-n/2}\int_{(0, \infty)} z^{(n-2)/2}\EXP{-z} \mathrm{d}z\\
    &= a^{-n/2} \, \Gamma(n/2)\\
    &\propto a^{-n/2}\\
    &= [(n-1)s^2 + n(\mu - \bar{y})^2]^{-n/2}\\
    &\propto \left[1 + \frac{1}{n-1} \frac{(\mu - \bar{y})^2}{s^2 / n}\right]^{-n/2} \,
  \end{align*}
  where $\Gamma$ denotes the gamma function (which is finite on the positive real numbers).
  This concludes the proof by implying that $\mu \mid y \sim t_{n-1}(\bar{y}, s^2/n)$,
\end{proof}


\begin{proof}[Proof of Proposition \ref{prop:posterior_conjugate}.]
Let us first state equation \ref{eq:conjugate_posterior} and the premise again.
We have to show that
\begin{align*}
  p(\mu, \sigma^2 \mid y) \propto& (\sigma^2)^{-\frac{3 + \nu_0 + n}{2}} \times\\
  & \times \EXP{-\frac{1}{2 \sigma^2} \left[\nu_0\sigma_0^2 + \kappa_0(\mu - \mu_0)^2 + (n-1)s^2 + n(\bar{y} - \mu)^2 \right]}
\end{align*}
is $\NormalscaledInvChi{\mu_n, \sigma_n^2/\kappa_n; \nu_n, \sigma_n^2}$ with $\nu_n = \nu_0 + n$, $\kappa_n = \kappa_0 + n$, $\mu_n =\frac{\kappa_0}{\kappa_0 + n}\mu_0 + \frac{n}{\kappa_0 + n}\bar{y}$, $\sigma_n^2 = \left[\nu_0 \sigma_0^2 + (n-1)s^2 + \frac{\kappa_0 n}{\kappa_0 + n} (\bar{y} - \mu_0)^2\right] /\nu_n$.
By definition of the normal-scaled-inverse-$\chi^2$ distribution $\nu_n = \nu_0 + n$ follows trivially.
Let us therefore consider the term in square brackets in the exponential.
We have to show that
\begin{align*}
  \left[\nu_0\sigma_0^2 + \kappa_0(\mu - \mu_0)^2 + (n-1)s^2 + n(\bar{y} - \mu)^2 \right] = \nu_n \sigma_n^2 + \kappa_n (\mu - \mu_n)^2 \,.
\end{align*}
Plugging in for $\sigma_n^2$ we get for the right-hand side
\begin{align*}
  \nu_n \sigma_n^2 + \kappa_n (\mu - \mu_n)^2 = \nu_0 \sigma_0^2 + (n-1)s^2 + \frac{\kappa_0 n}{\kappa_0 + n} (\bar{y} - \mu_0)^2 + \kappa_n (\mu- \mu_n)^2 \,.
\end{align*}
Therefore we only need to check
\begin{align*}
  \kappa_0(\mu - \mu_0)^2 + n(\bar{y} - \mu)^2 = \frac{\kappa_0 n}{\kappa_0 + n} (\bar{y} - \mu_0)^2 + \kappa_n (\mu- \mu_n)^2 \,.
\end{align*}
Expanding the right-hand side we get
\begin{align*}
\frac{\kappa_0 n}{\kappa_0 + n} &(\bar{y} - \mu_0)^2 + \kappa_n (\mu- \mu_n)^2\\
&=\frac{\kappa_0 n}{\kappa_n}\left[\bar{y}^2 - 2\bar{y}\mu_0 + \mu_0^2 \right] + \kappa_n \left[\mu^2 - 2\mu\mu_n + \mu_n^2 \right]\\
&=\frac{\kappa_0 n}{\kappa_n}\left[\bar{y}^2 - 2\bar{y}\mu_0 + \mu_0^2 \right] + \kappa_n \left[\mu^2 - 2\mu\frac{\kappa_0}{\kappa_n}\mu_0 - 2\mu\frac{n}{\kappa_n}\bar{y} + \frac{\kappa_0^2}{\kappa_n^2}\mu_0^2 + \frac{n^2}{\kappa_n^2} \bar{y}^2 + 2\frac{\kappa_0}{\kappa_n} \frac{n}{\kappa_n}\mu_0\bar{y} \right]\\
&=\frac{\kappa_0 n}{\kappa_n}\left[\bar{y}^2 - 2\bar{y}\mu_0 + \mu_0^2 \right] + \kappa_n \mu^2 - 2\mu \kappa_0 \mu_0 - 2\mu n\bar{y} + \frac{\kappa_0^2}{\kappa_n}\mu_0^2 + \frac{n^2}{\kappa_n} \bar{y}^2 + 2 \kappa_0 n \mu_0 \bar{y} / \kappa_n\\
&= \left(\kappa_0 \mu^2 -2\mu \kappa_0 \mu_0 \right) + \left(n \mu^2 - 2 n \mu \bar{y} \right) + \frac{\kappa_0 n}{\kappa_n}\bar{y}^2 + \frac{\kappa_0 n}{\kappa_n}\mu_0^2 + \frac{\kappa_0^2}{\kappa_n} \mu_0^2 + \frac{n^2}{\kappa_n} \bar{y}^2\\
&= \left(\kappa_0 \mu^2 -2\mu \kappa_0 \mu_0 \right) + \left(n \mu^2 - 2 n \mu \bar{y} \right) + \bar{y}^2 \left(\frac{\kappa_0 n}{\kappa_n} + \frac{n^2}{\kappa_n}\right) + \mu_0^2\left(\frac{\kappa_0 n}{\kappa_n} + \frac{\kappa_0^2}{\kappa_n}\right)\\
&= \left(\kappa_0 \mu^2 -2\mu \kappa_0 \mu_0 + \kappa_0 \mu_0^2 \right) + \left(n \mu^2 - 2 n \mu \bar{y} + n \bar{y}^2\right)\\
&=\kappa_0(\mu - \mu_0)^2 + n(\bar{y} - \mu)^2 \,,
\end{align*}
which was what we wanted.
\end{proof}

\begin{proof}[Proof of Proposition \ref{prop:marginal_posterior}.]
We continue to use the notation of the previous proof.
As in the proof of proposition \ref{prop:posterior_uniform} we first compute the distribution of $\mu \mid \sigma^2, y$ and then derive the posterior of $\sigma^2$ by integrating $\mu$ out.
Note that we actually defined $\mu \mid \sigma^2 \sim \normal{\mu_0, \sigma^2/\kappa_0}$.
Hence,
\begin{align*}
  p(\mu \mid \sigma^2, y) &\propto p(y \mid \mu, \sigma^2) p(\mu \mid \sigma^2)\\
  &\propto \EXP{-\frac{1}{2\sigma^2/n} (\mu - \bar{y})^2} \EXP{-\frac{1}{2\sigma^2/\kappa_0}(\mu - \mu_0)^2}\\
  &= \EXP{-\frac{1}{2\sigma^2}\left[n(\mu - \bar{y})^2 + \kappa_0(\mu - \mu_0)^2 \right]}\\
  &= \EXP{-\frac{1}{2\sigma^2}\left[\mu^2(\kappa_0 + n) - 2\mu(\kappa_0\mu_0 + n\bar{y}) + (\mydots) \right]}\\
  &= \EXP{-\frac{1}{2\sigma^2 / \kappa_n}\left[\mu^2 - 2\mu(\kappa_0\mu_0 + n\bar{y})/\kappa_n + (\mydots)/\kappa_n \right]}\\
  &= \EXP{-\frac{1}{2\sigma^2 / \kappa_n}\left(\mu - \mu_n^2\right) + (\mydots)}\\
  &\propto \EXP{-\frac{1}{2\sigma^2 / \kappa_n}\left(\mu - \mu_n^2\right)} \,,
\end{align*}
which implies that $\mu \mid \sigma^2, y \sim \normal{\mu_n, \sigma^2 / \kappa_n}$, where we used $(\mydots)$ to denote constants independent of $\mu$.

Now we can use this result as
\begin{align*}
  p(\sigma^2 \mid y) &= \int p(y, \sigma^2 \mid y) \mathrm{d}\mu\\
  &\propto \int (\sigma^2)^{-\frac{3 + \nu_n}{2}} \EXP{\frac{1}{2\sigma^2} \left[\nu_n \sigma_n^2 + \kappa_n(\mu_n - \mu)^2 \right]}\mathrm{d}\mu\\
  &\propto (\sigma^2)^{-\frac{3 + \nu_n}{2}} \int \EXP{\frac{1}{2\sigma^2} \nu_n \sigma_n^2}\EXP{\frac{1}{2\sigma^2 / \kappa_n} (\mu_n - \mu)^2 }\mathrm{d}\mu\\
  &\propto (\sigma^2)^{-\frac{3 + \nu_n}{2}}\EXP{\frac{1}{2\sigma^2} \nu_n \sigma_n^2} \int \EXP{\frac{1}{2\sigma^2 / \kappa_n} (\mu_n - \mu)^2 }\mathrm{d}\mu\\
  &\propto (\sigma^2)^{-\frac{3 + \nu_n}{2}}\EXP{\frac{1}{2\sigma^2} \nu_n \sigma_n^2} \sqrt{2 \pi \sigma^2 / \kappa_n}\\
  &\propto (\sigma^2)^{-(1 + \frac{\nu_n}{2})} \EXP{-\frac{1}{2\sigma^2}\nu_n\sigma_n^2}\,,
\end{align*}
from which we can conclude that $\sigma^2 \mid y \sim \scaledInvChi{\nu_n, \sigma_n^2}$.

We end the proof by deriving the marginal posterior of $\mu$ using an analogous approach as in the proof of Proposition \ref{prop:posterior_uniform}.
Define $a := \left[\nu_n\sigma_n^2 + \kappa_n(\mu_n - \mu)^2 \right]$. We solve for the posterior by integrating $\sigma^2$ out using the substitution $z = \frac{a}{2\sigma^2}$. Then
\begin{align*}
  p(\mu \mid y) &= \int_{(0, \infty)} p(\mu, \sigma^2 \mid y) \mathrm{d}\sigma^2\\
  &\propto \int_{(0, \infty)}(\sigma^2)^{-\frac{3 + \nu_n}{2}} \EXP{\frac{1}{2\sigma^2} \left[\nu_n \sigma_n^2 + \kappa_n(\mu_n - \mu)^2 \right]}\mathrm{d}\sigma^2\\
  &\propto \int_{(0, \infty)}(\sigma^2)^{-\frac{3 + \nu_n}{2}} \EXP{\frac{a}{2\sigma^2}}\mathrm{d}\sigma^2\\
  &\propto \int_{(0, \infty)}(a / 2z)^{-\frac{3 + \nu_n}{2}} \EXP{-z} \frac{a}{2 z^2} \mathrm{d}z\\
  &\propto \int_{(0, \infty)}a^{-\frac{3 + \nu_n}{2}}a z^{\frac{3 + \nu_n}{2}}z^{-2} \EXP{-z} \mathrm{d}z\\
  &= a^{-\frac{1 + \nu_n}{2}} \int_{(0, \infty)} z^{\frac{\nu_n - 1}{2}} \EXP{-z} \mathrm{d}z\\
  &= a^{-\frac{1 + \nu_n}{2}} \Gamma\left(\frac{\nu_n + 1}{2}\right)\\
  &\propto a^{-\frac{1 + \nu_n}{2}}\\
  &= \left[\nu_n \sigma_n^2 + \kappa_n(\mu_n - \mu)^2 \right]^{-\frac{1 + \nu_n}{2}}\\
  &= \left[\nu_n \sigma_n^2\left(1 + \frac{1}{\nu_n}\frac{(\mu_n - \mu)^2}{\sigma_n^2 / \kappa_n}\right) \right]^{-\frac{1 + \nu_n}{2}}\\
  &\propto \left[1 + \frac{1}{\nu_n}\frac{(\mu_n - \mu)^2}{\sigma_n^2 / \kappa_n} \right]^{-\frac{1 + \nu_n}{2}} \,,
\end{align*}
which concludes the proof by implying that $\mu \mid y \sim t_{\nu_n}(\mu_n, \sigma_n^2 / \kappa_n)$.
\end{proof}

%\begin{proof}[Proof of Proposition \ref{prop:hierarchical_posterior}.]
%  Let us first consider the parameter definitions again for clarity.
%  We define
%  \begin{align*}
%    \sigma_\mu^{-2} = \sum_j \frac{1}{\sigma_j^2 + \tau^2}\,,\,\, \tau_j^{-2}=\frac{1}{\sigma_j^2} + \frac{1}{\tau^2}\,,\,\, \mu_j=\tau_j^2(\frac{1}{\sigma_j^2} \bar{y}_j + \frac{1}{\tau^2}\mu) \,,\,\, \bar{\mu} = \sigma_\mu^2 \sum_j \frac{1}{\sigma_j^2 + \tau^2} \bar{y}_j \,.
%  \end{align*}
%  \noindent
%  \textbf{\emph{i.)}} Note that $p(\theta \mid \mu, \tau^2, y) \propto p(y \mid \theta, \mu, \tau^2) p(\theta \mid \mu, \tau^2) = p(y \mid \theta) p(\theta \mid \mu, \tau^2)$, as conditional on $\theta$ the observations and hyperparameter are independent.
%  Now by construction we get $p(\theta \mid \mu, \tau^2) = \prod_j p(\theta_j \mid \mu, \tau^2)$.
%  Note that since we assume independence we clearly also get $p(\theta \mid \mu, \tau^2, y) = \prod_j p(\theta_j \mid \mu, \tau^2, y)$; however to get the distribution of $p(\theta_j \mid \mu, \tau^2, y)$ we split the likelihood into factors dependent only on $\theta_j$ and show that the product of the factors constitutes a normal density with the desired parameters.
%  So we have to show that $p(y \mid \theta)$, considered as a function of $\theta$, factors into $J$ factors each only dependent on $\theta_j$.
%  Denoting terms not dependent on the $\theta_j's$ by $(\mydots)$, we get
%  \begin{align*}
%    p(y \mid \theta) &= \prod_{i=1}^n p(y_i \mid \theta)=\prod_{i=1}^n p(y_i \mid \theta_{j[i]}) = \prod_{j=1}^J \prod_{i:j[i]=j} p(y_i \mid \theta_{j})\\
%    &= \prod_j (2 \pi \sigma^2)^{-n_j / 2} \EXP{-\frac{1}{2\sigma^2}\sum_{i:j[i]=j} (y_i - \theta_j)^2}\\
%    &\propto \prod_j \EXP{-\frac{1}{2\sigma^2} \left[n_j \theta_j^2 - n_j 2\theta_j\bar{y}_j + (\mydots)\right]}\\
%    &\propto \prod_j \EXP{-\frac{1}{2\sigma^2 / n_j} (\theta_j - \bar{y}_j)^2} \,.
%  \end{align*}
%  Hence,
%  \begin{align*}
%    p(\theta \mid \mu, \tau^2, y) &\propto p(y \mid \theta) p(\theta \mid \mu, \tau^2)\\
%    &\propto \prod_j \EXP{-\frac{1}{2\sigma^2 / n_j} (\theta_j - \bar{y}_j)^2} \prod_j p(\theta_j \mid \mu, \tau^2)\\
%    &\propto \prod_j \EXP{-\frac{1}{2\sigma^2 / n_j} (\theta_j - \bar{y}_j)^2} \EXP{-\frac{1}{2\tau^2}(\theta_j - \mu)^2}\\
%    &= \prod_j \EXP{-\frac{1}{2\sigma^2 / n_j} (\theta_j - \bar{y}_j)^2 -\frac{1}{2\tau^2}(\theta_j - \mu)^2}\\
%    &= \prod_j \EXP{-\frac{1}{2}\left[\left(\frac{1}{\sigma^2/n_j} + \frac{1}{\tau^2}\right)\theta_j^2 - 2\theta_j\left(\frac{1}{\sigma^2/n_j}\bar{y}_j + \frac{1}{\tau^2}\mu\right) + (\mydots) \right]}\\
%    &= \prod_j \EXP{-\frac{1}{2}\left[\tau_j^{-2} \theta_j^2 - 2\theta_j\mu_j\tau_j^{-2} + (\mydots) \right]}\\
%    &\propto \prod_j \EXP{-\frac{1}{2\tau_j^2} (\theta_j - \mu_j)^2} \,,
%  \end{align*}
%which gives $\theta_j \mid \mu, \tau^2, y \sim \normal{\mu_j, \tau_j^2}$.\\
%
%  \noindent
%  \textbf{\emph{ii.)}} Let us first prove an intermediate result, namely $\bar{y}_j \mid \mu, \tau^2 \sim \normal{\mu, \sigma_j^2 + \tau^2}$ for all $j$.
%  Note that we may write
%  \begin{align*}
%    p(\bar{y}_j \mid \mu, \tau^2) &= \int p(\bar{y}_j, \theta_j \mid \mu, \tau^2)\mathrm{d}\theta_j = \int p(\bar{y}_j \mid \theta_j) p(\theta_j \mid \mu, \tau^2)\mathrm{d}\theta_j \,.
%  \end{align*}
%  We know that $\bar{y}_j \mid \theta_j \sim \normal{\theta_j, \sigma_j^2}$ and $\theta_j \mid \mu, \tau^2 \sim \normal{\mu, \tau^2}$, which means that we can factor the term in the integral into an exponential containing a quadratic term in $\bar{y}_j$ (and not $\theta_j$) and an expontential containing a quadratic term in $\theta_j$.
%  The former term we may pull out of the intgral and the latter disappears into a constant (not depending on $\bar{y}_j$) as we integrate over $\theta_j$.
%  We omit a detailed derivation here and refer to \citet{gelmanbda04}.
%
%  Then using that for $j=1,\mydots,J$ the quantities $\bar{y}_j \mid \mu, \tau^2$ are independent, we can write
%  \begin{align*}
%    p(\mu, \tau^2 \mid y) \propto p(y \mid \mu, \tau^2) p(\mu, \tau) = \prod_j p(\bar{y}_j \mid \mu, \tau^2) p(\mu \mid \tau) p(\tau) \propto p(\tau) \prod_j p(\bar{y}_j \mid \mu, \tau^2) \,,
%  \end{align*}
%  where we use that $p(\mu \mid \tau) \propto 1$.
%  Note that we can also write $p(\mu, \tau \mid y) = p(\mu \mid \tau, y) p(\tau \mid y)$.
%  Therefore we get
%  \begin{align*}
%    p(\mu \mid \tau, y) p(\tau \mid y) \propto p(\tau) \prod_j p(\bar{y}_j \mid \mu, \tau^2) \,.
%  \end{align*}
%  Now if we consider the factors depending on $\mu$ more closely we see that (using equivalent transformations as above and plugging in the definitions)
%  \begin{align*}
%    \prod_j p(\bar{y}_j \mid \mu, \tau^2) &\propto \prod_j \EXP{-\frac{1}{2(\sigma_j^2 + \tau^2)}(\bar{y}_j - \mu)^2}\\
%    &=\EXP{-\frac{1}{2}\sum_j \frac{1}{\sigma_j^2 + \tau^2}(\bar{y}_j - \mu)^2}\\
%    &\propto \EXP{-\frac{1}{2 \sigma_\mu^2} (\mu - \bar{\mu})^2}
%  \end{align*}
%  where proportionality is with respect to $\mu$.
%  As we can factor $p(\mu \mid \tau^2, y)$ and as $\mu$ only appears as a quadratic in the exponential we may derive that $\mu \mid \tau^2, y \sim \normal{\bar{\mu}, \sigma_\mu^2}$, which was what we wanted.
%
%  \noindent
%  \textbf{\emph{iii.)}}
%
%\end{proof}



\end{document}
