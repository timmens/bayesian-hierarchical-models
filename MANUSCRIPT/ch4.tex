\section{Application}

In this section we discuss the applications of the hierarchical modelling approach in the form of a literature review. We then use real world data to give a practical illustration of the work we have developed in the previous sections of our paper and compare that to the frequentist approach to analyzing multilevel data with the aim of showcasing what differences/similarities there are between the two methods. We then depart form the comparative analysis to look deeper into the Bayesian approach and conduct robustness checks that vary certain aspects (e.g. priors) of it so that we can see how the estimation results change in response to that. The baseline here would be the estimation results form the comparison between Bayesian and frequentist approaches. We close the section off by a discussion of the limitations and challenges encountered in this section.


\subsection{Literature Review on the application of Hierarchical Models in Economics}

Hierarchical models are a suitable approach to consider social contexts as well as individual respondents or subjects. It becomes attractive to consider hierarchical models in place of the common (or popularized) frequentist approach as soon as there is a need to relax the independence of residuals assumption as a result of similarities in the characteristics of a group of respondents or when the researcher seeks to disentangle variability at various levels of the data. 

\subsection{Frequentist and Bayesian Approaches in Practice:}

\subsection{Looking Deeper into the Baysian Approach: Robustness Checks}

\subsection{Challenges and Limitations}






