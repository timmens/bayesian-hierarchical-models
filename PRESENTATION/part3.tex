\section{Monte Carlo Study}


\begin{frame}{Stan}
  \Large{

  \begin{itemize}
      \item[] \emphcol{What is Stan?} C\texttt{++} package (fast run times) \pause
      \item[] \emphcol{Use cases:} Bayesian/Maximum Likelihood estimation of statistical models\pause
      \item[] \emphcol{Interfaces:} Pystan, Rstan,Stan.jl,...
   \end{itemize}
  }
\end{frame}

\begin{frame}{Bayesian Models with Stan}
  \Large{
  Stan uses an internal modelling language for model specification. \pause
  We tested the Random Slope model:
 \begin{align}
  y = \alpha + \beta_j x + \epsilon,  \epsilon \sim N(1,1) \pause
 \end{align}
 \begin{align}
  \beta_j = \gamma_0 + \gamma_1 u_j + \eta_j , \eta  \sim N(1,1)
 \end{align}
}
  \begin{figure}
  \centering
  \includegraphics<3>[height=2.8 cm]{graphics/stan_data} \pause
  \includegraphics<4>[height=2.8 cm]{graphics/stan_parameter} \pause
  \includegraphics<5>[height=2.8 cm]{graphics/stan_model} \pause
  \includegraphics<6>[height=2.8 cm]{graphics/stan_priors} 
  \end{figure}
\end{frame}

\begin{frame}{How can we sure that we sample from the right distribution? }
  \Large{
  %\vfill	
  \begin{figure}
  \centering
  \includegraphics<1>[height=5.3 cm]{graphics/single-chain-1} \pause
   \includegraphics<2>[height=5.3 cm]{graphics/single-chain-2}\pause
    \includegraphics<3>[height=5.3 cm]{graphics/single-chain-3}
  \end{figure}
  %\vfill

  \begin{itemize}
     \item[] Use a number of burn-in draws
 \end{itemize}
  }
\end{frame}



\begin{frame}{Monitoring Convergence }
  \Large{
  %\vfill	
  \begin{figure}
  \centering
  \includegraphics<1>[height=5.3 cm]{graphics/s-convergence} \pause
   \includegraphics<2->[height=5.3 cm]{graphics/m-convergence}
  \end{figure}
  %\vfill
  }
  \begin{itemize}
     \only <1>    { \item[] Quantify this approach}
     \only<2-3> { \item[] \emphcol{Variance of a single chain:} $$ s_m^2= \frac{1}{N-1}\sum_{n=1}^{N} (\theta_m^{(n)}-\overline{\theta}_{m}^{(\bullet)} )^2$$ }\pause
 \item[] \emphcol{Average within chain variance:} $$W=\frac{1}{M} \sum_{m=1}^{M} s_m^2 $$  
 \end{itemize}
\end{frame}



\begin{frame}{Monitoring Convergence }
  \Large{
  %\vfill	
  \begin{figure}
  \centering
  \includegraphics<1>[height=5.3 cm]{graphics/s-convergence} \pause
   \includegraphics<2->[height=5.3 cm]{graphics/m-convergence}
  \end{figure}
  %\vfill
  }
  \begin{itemize}
     \only <1>    { \item[] Quantify this approach}
     \only<2-3> { \item[] \emphcol{Variance of a single chain:} $$ s_m^2= \frac{1}{N-1}\sum_{n=1}^{N} (\theta_m^{(n)}-\overline{\theta}_{m}^{(\bullet)} )^2$$ }\pause
 \item[] \emphcol{Average within chain variance:} $$W=\frac{1}{M} \sum_{m=1}^{M} s_m^2 $$  
 \end{itemize}
\end{frame}

\begin{frame}{Brooks and Gelman convergence criterium}
  \Large{

  \begin{itemize}
       \item[] \emphcol{Average Variance between chains:} $$ B/N=\frac{1}{M-1} \sum_{m=1}^{M} (\overline{\theta}_m^{(\bullet)}  - \overline{\theta}_{\bullet}^{(\bullet)} )^2 $$  \pause
 \item[] \emphcol{Total Variance:} $$\widehat{var}^+ (\theta \mid y)=\frac{N-1}{N}W+\frac{1}{N}B$$   

\end{itemize}
  }
\end{frame}

\begin{frame}{Monitoring Convergence}
  \Large{
  %\vfill
  \begin{itemize}
     \item[] \emphcol{Scale Reducing Factor:} $$\widehat{R}=\sqrt{\frac{\widehat{var}^+ (\theta \mid y)}{W}}$$  \pause
 \end{itemize}
	
  \begin{figure}
  \centering
  \includegraphics<2>[height=5.5cm]{graphics/failure-convergence} \pause
  \includegraphics<3>[height=5.5cm]{graphics/sucess-convergence}
  \end{figure}
  %\vfill
  }
\end{frame}

\begin{frame}{Study Design}
 \begin{itemize}
 \item[] \emphcol{Good Prior:}   $\gamma_0 \sim N(1,1),  \gamma_1 \sim N(1,1)$ \pause
 \item[] \emphcol{Bad Prior:}   $\gamma_0 \sim N(2,1),  \gamma_1 \sim N(2,1)$ \pause
 \item[] \emphcol{Bad weak Prior:}   $\gamma_0 \sim N(2,3),  \gamma_1 \sim N(2,3)$ \pause
 \item[] \emphcol{Univariate Prior:}   $\gamma_0 \sim,  UNI[-\infty,\infty] ,  \gamma_1 \sim UNI[-\infty,\infty]$ \pause

 \item[] \emphcol{In all models:}  $\sigma_y, \sigma_b \sim$ half-cauchy (0,5)   
 \end{itemize}


\end{frame}

     



\begin{frame}{Posterior Distribution with a good prior }
  \Large{	
  \begin{figure}
  \centering
  \includegraphics<1>[height=5.5cm]{graphics/fitting-posterior} 
 \caption{Posterior Draws of $\gamma_1$ with N=200, J=10 and 300 simulations}
  \end{figure}
  %\vfill
  }
\end{frame}


\begin{frame}{What happense if we decrease the number of levels J? }
  \Large{	
  \begin{figure}
  \centering
  \includegraphics<1>[height=5.5cm]{graphics/fitting-posterior-small} 
 \caption{Posterior Draws of $\gamma_1$ with N=50, J=5 and 300 simulations}
  \end{figure}
  %\vfill
  }
\end{frame}


\begin{frame}{Posterior Distribution with a bad prior }
  \Large{	
  \begin{figure}
  \centering
  \includegraphics<1>[height=5.5cm]{graphics/nonfitting-posterior} 
 \caption{Posterior Draws of $\gamma_1$ with N=200, J=10 and 300 simulations}
  \end{figure}
  }
\end{frame}

\begin{frame}{Posterior Distribution for a bad, but weak prior }
  \Large{	
  \begin{figure}
  \centering
  \includegraphics<1>[height=5.5cm]{graphics/partfitting-posterior} 
 \caption{Posterior Draws of $\gamma_1$ with N=200, J=10 and 150 simulations}
  \end{figure}
  }
\end{frame}


\begin{frame}{Not a good idea: Uniform Prior}
  \Large{
  \begin{figure}
  \centering
  \includegraphics<1>[height=5.5cm]{graphics/uni-posterior} 
\caption{Posterior Draws of $\gamma_1$ with N=200, J=10 and 300 simulations}  
\end{figure}
  }
\end{frame}


\begin{frame}{Increase sample size dramatically}
  \Large{
  %\vfill
	
  \begin{figure}
  \centering
  \includegraphics<1>[height=5.5cm]{graphics/posterior-big} 
 \caption{Single posterior draw for model with wrong prior and N=500, J=50}

  \end{figure}
  }
\end{frame}

